\documentclass[a4paper]{amsart}

\usepackage[T1]{fontenc}
\usepackage[utf8]{inputenc}
\usepackage[english]{babel}
%\usepackage[brazil]{babel}		%portuguese hyphenation
\usepackage[includeheadfoot, margin=1in]{geometry}

\usepackage[bookmarksnumbered]{hyperref}    %Adds ability to crete hyperlinks

\usepackage{amsmath}
\usepackage{amssymb}
\usepackage{mathrsfs}
\usepackage{eucal}			%\CMcal for original cal
%\usepackage{nicefrac}  %nice fractions: \nicefrak[<fontcmd>]{<num>}{<den>}
\usepackage{tensor} %Better tensor indices \tensor[left]{symbol}{right}. \tensor* collapses spacing

%\usepackage{wrapfig}			%figures surrounded by text
%\usepackage{graphicx}			%figures
%\usepackage{multicol}		%Multicolumn environment, \begin{multicols}{# of columns}\end{multicols}

\usepackage{amsthm}
\usepackage{enumitem}  %Options for enumerate environment
\usepackage[normalem]{ulem} %Better underlines

%\everymath{\displaystyle}
%\usepackage{tikz-cd}			%\usetikzlibrary{matrix}	%diagramas
%\usepackage{tikz}  \usetikzlibrary{arrows}
%\usepackage[vcentermath]{youngtab}			%Young diagrams and tableaux

%\usepackage{tabularx}  %auto-sized columns in a fixed width table
%\usepackage{ltablex}   %combine tabularx with package longtable to obtain multi-page tables

%\usepackage[numbers]{natbib}    %bibtex
%\usepackage{doi}

\usepackage[backend=biber,style=ieee,url=false]{biblatex} %biber
\bibliography{library.bib}
\usepackage{csquotes}

%%%%%%%%%%%%%%%%%%%%%%%%%%%%%%%%%%%%%%%
%%%%%%%%%%%%%%%%%%%%%%%%%%%%%%%%%%%%%%%
%%%%%%%%%%%%%%%%%%%%%%%%%%%%%%%%%%%%%%%
%ENVIRONMENTS
%%%%%%%%%%%%%%%%%%%%%%%%%%%%%%%%%%%%%%%

\newcounter{generalnumbering}   \numberwithin{generalnumbering}{section}

\theoremstyle{plain}    \newtheorem{theorem}[generalnumbering]{Theorem}
\theoremstyle{plain}    \newtheorem{corollary}[generalnumbering]{Corollary}
\theoremstyle{definition}   \newtheorem{definition}[generalnumbering]{Definition}
\theoremstyle{definition}   \newtheorem{example}[generalnumbering]{Example}
\theoremstyle{plain}    \newtheorem{proposition}[generalnumbering]{Proposition}
\theoremstyle{plain}    \newtheorem{lemma}[generalnumbering]{Lemma}

\newcommand{\namefordifferentenvironment}{}

\theoremstyle{plain}    \newtheorem{plainstyle}[generalnumbering]{\namefordifferentenvironment}
\theoremstyle{plain}    \newtheorem*{plainstyle*}{\namefordifferentenvironment}
\theoremstyle{definition}    \newtheorem{definitionstyle}[generalnumbering]{\namefordifferentenvironment}
\theoremstyle{definition}    \newtheorem*{definitionstyle*}{\namefordifferentenvironment}

\newenvironment{penv}[1]{\renewcommand{\namefordifferentenvironment}{#1}\begin{plainstyle}}{\end{plainstyle}}
\newenvironment{penv*}[1]{\renewcommand{\namefordifferentenvironment}{#1}\begin{plainstyle*}}{\end{plainstyle*}}
\newenvironment{denv}[1]{\renewcommand{\namefordifferentenvironment}{#1}\begin{definitionstyle}}{\end{definitionstyle}}
\newenvironment{denv*}[1]{\renewcommand{\namefordifferentenvironment}{#1}\begin{definitionstyle*}}{\end{definitionstyle*}}

\newcommand{\ntag}{\tag{\thegeneralnumbering}\stepcounter{generalnumbering}}

%%%%%%%%%%%%%%%%%%%%%%%%%%%%%%%%%%%%%%%
%%%%%%%%%%%%%%%%%%%%%%%%%%%%%%%%%%%%%%%
%%%%%%%%%%%%%%%%%%%%%%%%%%%%%%%%%%%%%%%
%New commands
%%%%%%%%%%%%%%%%%%%%%%%%%%%%%%%%%%%%%%%

\renewcommand*{\thefootnote}{[\arabic{footnote}]}   %Footnote numbers with brackets

\newcommand{\perpp}{\perp\!\!\!\perp}
\DeclareMathOperator{\supp}{supp}
\DeclareMathOperator{\id}{id}
\DeclareMathOperator{\dom}{dom}
\DeclareMathOperator{\ran}{ran}

\newcommand{\cat}[1]{\normalfont{\textsc{\textbf{#1}}}}

\newcommand*{\defeq}{\mathrel{\vcenter{\baselineskip0.5ex \lineskiplimit0pt\hbox{\scriptsize.}\hbox{\scriptsize.}}}=}
\newcommand*{\eqdef}{=\mathrel{\vcenter{\baselineskip0.5ex \lineskiplimit0pt\hbox{\scriptsize.}\hbox{\scriptsize.}}}}

%%%%%%%%%%%%%%%%%%%%%%%%%%%%%%%%%%%%%%%
%%%%%%%%%%%%%%%%%%%%%%%%%%%%%%%%%%%%%%%
%%%%%%%%%%%%%%%%%%%%%%%%%%%%%%%%%%%%%%%
%AUTHOR INFO
%%%%%%%%%%%%%%%%%%%%%%%%%%%%%%%%%%%%%%%

\title{The Title}
%\subtitle{}
%\date{}
\author{Luiz Gustavo Cordeiro%\thanks{Thanks 1} \thanks{Thanks 2} \thanks{Thanks 3} \thanks{Thanks 4}
}
%Thanks should be given separately for AMSART
%\thanks{Thanks 1}
%\thanks{Thanks 2}

%\address{UMPA, UMR 5669 CNRS -- École Normale Supérieure de Lyon\\46 allée d'Italie, 69364 Lyon Cedex 07, France}
%\email{luizgc6@gmail.com}				%AMSART


%\address{Institute of Mathematics and Statistics,	%AMSART
%University of Ottawa,
%585 King Edward Ave.,
%Ottawa, ON K1N 6N5,
%Canada}
%\email{luizgc6@gmail.com, luis-gustavo.cordeiro@ens-lyon.fr}				%AMSART

%\subjclass[2010]%		%AMSART
%{Primary ;
%Secondary
%,
%
%}

%\keywords{}		%AMSART

\begin{document}
\maketitle

\section{Espaços Vetoriais}

\begin{definition}
Um \emph{espaço vetorial} consiste de um conjunto não vazio $V$, cujos elementos são chamados de \emph{vetores}, munido de duas operações:
\begin{itemize}
    \item A \emph{soma} ou \emph{adição} de vetores, que a cada par de vetores $u,v$ faz corresponder um novo vetor $u+v$, chamado de \emph{soma} de $u$ e $v$.
    \item A \emph{multiplicação escalar}, que a cada vetor $v$ e a cada número real $\alpha$ associa um novo vetor $\alpha v$, chamado de \emph{produto} de $\alpha$ e $v$.
\end{itemize}
    Estas operações também devem satisfazer às seguintes condições abaixo (chamadas de \emph{axiomas} de espaços vetoriais): Para quaisquer vetores $u,v,w$ e quaisquer números reais $\alpha,\beta$:
    \begin{enumerate}[label=(\roman*)]
        \item \uline{Comutatividade}: $u+v=v+u$;
        \item \uline{Associatividade}: $(u+v)+w=u+(v+w)$ e $\alpha(\beta v)=(\alpha\beta)v$;
        \item \uline{Existência de vetor nulo}: Existe um vetor $0$ tal que $v+0=0+v=v$ para todo $v\in E$;
        \item \uline{Existência de inverso aditivo}: Para todo vetor $v$, existe um vetor $-v$, chamado de \emph{oposto} de $v$, tal que $v+(-v)=0$;
        \item \uline{Distributividade}: $(\alpha+\beta)v=\alpha v+\beta v$, e $\alpha(u+v)=\alpha u+\alpha v$;
        \item \uline{Multiplicação por $1$}: $1v=v$.
    \end{enumerate}
\end{definition}

\begin{denv*}{Observação}
    Note que o mesmo símbolo ``$0$'' é utilizado para denotar tanto o vetor nulo, quanto o número real zero.
\end{denv*}

Várias dessas propriedades são similares às satisfeitas pelas operações entre números reais. Porém, agora estamos trabalhando com dois conjuntos diferentes ao mesmo tempo -- $V$ e $\mathbb{R}$ -- que operam entre si, e devemos tomar cuidado com o modo com que as operações são tomadas. Por exemplo, podemos multiplicar dois números reais, ou um número real e um vetor; mas não podemos multiplicar dois vetores.

Para realçar a natureza distinta de vetores e de números, é comum utilizar-se de convenções ortográficas; Por exemplo, às vezes se convenciona que vetores são denotados por letras com setas em cima (como em ``$\vec{w}$'').

Na convenção que utilizaremos, as letras finais do alfabeto ($u$, $v$, $w$,\ldots) geralmente denotarão vetores, e letras gregas ($\alpha$, $\beta$, $\gamma$,\ldots) denotarão números reais.

\begin{example}
    Considere o conjunto $V=[0,\infty)]$ dos números reais não-negativos. Vamos definir as seguintes operações $\oplus$ e $\odot$ em $V$: Para quaisquer $v,w\in V$ e qualquer $\alpha\in\mathbb{R}$,
    \[v\oplus w=v+w\quad\text{e}\quad\alpha \odot v=|\alpha|v,\]
    (onde as operações nos lados direitos das equações são a soma e produto usuais de $\mathbb{R}$).
    
    Então $V$, com as operações $\oplus$ e $\odot$, satisfaz a alguns dos axiomas de espaços vetoriais, como à propriedade comutativa. Porém, $V$ não satisfaz à propriedade distributiva: Se tomarmos $\alpha=1$, $\beta=-1$, e $v=1$, então
    \[(\alpha+\beta)\odot v=|1-1|\cdot 1=0\cdot 1=0;\]
    mas
    \[(\alpha\odot v)\oplus(\beta\odot v)=(|1|\cdot 1)+(|1|\cdot 1)=1+1=2,\]
    e portanto os termos no lado esquerdo são diferentes.
    
    Por um lado, os axiomas de espaços vetoriais devem ser satisfeitos para \uline{todas as escolhas} de vetores e números reais. Mas como pudemos fazer uma escolha que não satisfaz à uma das propriedades, isto significa que $V$ não é um espaço vetorial.
\end{example}

\begin{example}
    O conjunto $\mathbb{R}$ de números reais, com sua soma e multiplicação usuais, é um espaço vetorial.
\end{example}

%\bibliographystyle{amsplain}   5Just for bibtex
%\bibliography{library}

\printbibliography  %Biber

\end{document}