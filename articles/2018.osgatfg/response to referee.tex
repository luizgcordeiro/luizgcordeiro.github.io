\documentclass[11pt]{amsart}

\setlength{\textwidth}{125mm}
\setlength{\textheight}{195 mm}
%\usepackage{anyfontsize}		%mais tamanhos de fonte
\usepackage{fullpage}			%margens menores
%\usepackage{wrapfig}			%poder pôr figuras no meio do texto
%\usepackage{graphicx}			%poder inserir figuras
%\usepackage[portuges]{babel}		%hifenização em português
\usepackage[english]{babel}
\usepackage[utf8]{inputenc}		%poder escrever acentos no código
\usepackage[T1]{fontenc}		%uma fonte mais completa ao compilar (e.g. acentos sem isso são símbolos isolados)
						%http://tex.stackexchange.com/questions/664/why-should-i-use-usepackaget1fontenc
\usepackage{amsmath}			%símbolos matemáticos
\usepackage{amssymb}			%mais símbolos matemáticos
\usepackage{amsthm}			%poder usar 'theoremstyle'
\usepackage{mathrsfs}			%\mathscr
%\usepackage[pdftex]{hyperref}		%transformar \ref em links
\usepackage[pdftex,bookmarksnumbered]{hyperref}	%números no índice do leitor de PDF
%\usepackage{nicefrac}			%frações legais - \nicefrac[<fontcmd>]{<num>}{<den>}
\usepackage{eucal}
\usepackage{amsfonts}
\usepackage{multicol}

%\everymath{\displaystyle}		%todo texto matematico em displaystyle
\usepackage{tikz-cd}			%diagramas
%\usepackage[vcentermath]{youngtab}			%Young diagrams and tableaux

%\usepackage[initials]{amsrefs}
\usepackage[numbers]{natbib}

\hyphenation{fe-cha-do res-tri-ta e-xis-tem con-ver-gen-tes con-ver-gen-te e-qui-va-len-tes li-mi-ta-do li-mi-ta-da sub-es-pa-ço sub-es-pa-ços de-mons-tra-ção de-mons-tra-ções ba-lan-ce-a-da ba-lan-ce-a-do de-mons-tra-do te-nha-mos to-po-ló-gi-co re-pa-ra-me-tri-za-ções re-pa-ra-me-tri-za-ção pa-ra-me-tri-za-das to-po-ló-gi-cos to-po-ló-gi-ca fun-çõe fun-ção so-lu-ções}

\newcounter{generalnumbering} \numberwithin{generalnumbering}{section}

\theoremstyle{plain}		\newtheorem{theorem}[generalnumbering]{Theorem}
\theoremstyle{plain}		\newtheorem{corollary}[generalnumbering]{Corollary}
\theoremstyle{definition}		\newtheorem{definition}[generalnumbering]{Definition}
\theoremstyle{definition}		\newtheorem{example}[generalnumbering]{Example}
\theoremstyle{plain}		\newtheorem{proposition}[generalnumbering]{Proposition}
\theoremstyle{plain}		\newtheorem{lemma}[generalnumbering]{Lemma}

\newenvironment{remark}
{\vspace{\topsep}\noindent\textbf{Remark.}}
{\vspace{\topsep}}

\author{Luiz Cordeiro}
%\thanks{$\hspace{0pt}^\dagger$Supported by CAPES/Ciência Sem Fronteiras PhD scholarship 012035/2013-00.}
\email{lcord081@uottawa.ca}
\address{University of Ottawa, Canada}

\title{Response to referee\\On sofic groupoids and their full groups\\Sur les groupoïdes sofiques et leur groupes pleins}

\begin{document}

\maketitle

I agree with all of the comments made by the referee, and they have all been addressed. Below are the changes I made:
\begin{enumerate}
\item[(1)] Stated instead that they are "injections and thus Borel isomorphisms onto their images", with appropriate reference.
\item[(2)] Added the definition of strong limit of measures as a footnote.
\item[(3)] Added a small paragraph after Definition 3.1, as an example ($G$ sofic implies $[G]$ metrically sofic). Added a paragraph at the beggining of the proof of 3.5 referring to it.
\item[(4)] Added ``sofic approximation'' to Definition 3.1.
\item[(5)] substituted by "the distance between $\phi_k(r(\alpha A))$ and $\theta_k(\alpha)(\phi(A))$ converges to zero". Also restated that $(\theta,\phi)$ is "asymptotically covariant" (instead of "invariant").
\item[(6)] Corrected.
\end{enumerate}

I was not aware of \cite{MR3229595}, and I thank the referee for pointing this out. A remark was added after Definition 2.2, and the abstract was modified accordingly (removed the parts stating we introduce this notion).

I also corrected the name in reference \cite{MR2826401}, and re-referenced the Lusin-Novikov theorem to Kechris' book \cite{MR1321597}), since the proof from Srivastava's book (\cite{MR1619545}) comes from Kechris'. I also corrected a typo in Example 1.2 (The inverse graph should be $\left\{(f(x),x):{\underline{x}}\in X\right\}$ and not $\left\{(f(x),x):{\underline{f}}\in X\right\}$).

Finally, I added a paragraph with "Acknowledgements" at the end, and moved the reference of my supervisors there instead of the footnote in the first page.

Below is the list of comments sent by the referee, for simplicity:

\begin{enumerate}
\item[(1)] p. 1, line 50: in the definition of the Borel full semigroup, it
would be more clear to say that $s|_A$ and $r|_A$ are both injections (as
opposed to bijections), since only the domains of these functions are
specified, but not the codomains.
\item[(2)] p. 3, line 15: It would be helpful if the definition of "strong
limit" of measures was given.
\item[(3)] p. 7, Theorem 3.5: Only one of the implications in the statement
of Theorem 3.5 is acknowledged in the proof sketch.
\item[(4)] p. 7, line 14: "sofic approximation for $[G]$" was never defined.
Presumably this means "a sequence of maps as in Definition 3.1," but
this should be made explicit
\item[(5)] p. 7, lines 18-19: "as close as necessary" is an awkward phrase
to use here and should be rephrased.
\item[(6)] p. 7, line 20: "sofic approximation of $[[G]]$" should be "sofic
approximation of G"
\end{enumerate}

\bibliographystyle{abbrv}
\bibliography{biblio}

\end{document}