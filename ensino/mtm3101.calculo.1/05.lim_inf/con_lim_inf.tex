\begin{definition}
	Sejam $f$ uma função real e $a$ um número real.

	Dizemos que a função $f$ tende a infinito positivo (ou mais infinito) em $a$, se para todo número real $A$, existe $\delta>0$ (dependendo de $A$) tal que se $x$ pertence ao domínio de $f$ e $0<|x-a|<\delta$, então $f(x)>A$. Neste caso, denotamos
	\[\lim_{x\to a}f(x)=+\infty.\]
	
	\hrule

	Dizemos que a função $f$ tende a infinito negativo (ou menos infinito) em $a$, se para todo número real $B$, existe $\delta>0$ (dependendo de $B$) tal que se $x$ pertence ao domínio de $f$ e $0<|x-a|<\delta$, então $f(x)<B$. Neste caso, denotamos
	\[\lim_{x\to a}f(x)=-\infty.\]
	
	\hrule

	Limites laterais infinitos são definidos similarmente, bem como limites infinitos no infinito. Estes são denotados pelas expressões $\displaystyle \lim_{x\to a^\pm}f(x)=\pm\infty$ e $\displaystyle \lim_{x\to\pm\infty}f(x)=\pm\infty$, conforme apropriado.
\end{definition}

\textit{Observação}: Caso $\lim_{x\to a}f(x)=\infty$, então o limite de $f$ no ponto $a$ não existe, pois $\pm\infty$ não é um número.