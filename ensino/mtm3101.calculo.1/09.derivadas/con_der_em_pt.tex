\begin{definition}
	Sejam $f$ uma função e $a$ um ponto de seu domínio. Dizemos que a função $f$ é \textbf{derivável} ou \textbf{diferenciável no ponto $a$} se existir o limite
	\[f'(a)=\lim_{x\to a}\dfrac{f(x)-f(a)}{x-a}.\]
	Neste caso, o valor $f'(a)$ é chamado de derivada de $f$ no ponto $a$.

	\hrule
	
	Algumas notações alternativas para derivada de $f$ no ponto $a$ são $Df(a)$ e $\dot{f}(a)$.

	Se a função $f$ estiver sendo escrita com uma variável de modo explítico -- do modo ``$f=f(x)$'' -- podemos integrar esta variável na notação da derivada, e denotá-la por $\left.\dfrac{df}{dx}\right|_a$, ou $\dfrac{df}{dx}(a)$, ou $D_xf(a)$.

	\hrule
	
	Note que fazendo-se a troca de variáveis $h=x-a$ na definição da derivada, obtemos a fórmula alternativa
	\[f'(a)=\lim_{h\to 0}\dfrac{f(a+h)-f(a)}{h}.\]
\end{definition}