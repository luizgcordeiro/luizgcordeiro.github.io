\begin{theorem}
	Se uma função $f$ é derivável em um ponto $a$ de seu domínio, então $f$ é contínua no ponto $a$.
	
	Em particular, se $f$ é derivável em todo o seu domínio, então $f$ é contínua.
\end{theorem}

\begin{proof}
	Temos os seguintes limites:
	\[\lim_{x\to a}\frac{f(x)-f(a)}{x-a}=f'(a)\]
	e
	\[\lim_{x\to a}x-a=0.\]
	
	Além disso, para $x\neq a$ temos que
	\[f(x)-f(a)=\left(\frac{f(x)-f(a)}{x-a}\right)(x-a).\]
	
	Portanto, o limite de $f(x)-f(a)$ quando $x$ tende a $a$ existe, e é dado por
	\begin{align*}
		\lim_{x\to a}\left(f(x)-f(a)\right)
			&=\left(\lim_{x\to a}\frac{f(x)-f(a)}{x-a}\right)\left(\lim_{x\to a}(x-a)\right)\\
			&=f'(a)\cdot 0\\
			&=0.
	\end{align*}
	
	Obtemos $\lim_{x\to a}f(x)-f(a)=0$. Isto é equivalente a dizer que $\lim_{x\to a}f(x)=f(a)$, o que significa precisamente que $f$ é contínua em $a$.
\end{proof}