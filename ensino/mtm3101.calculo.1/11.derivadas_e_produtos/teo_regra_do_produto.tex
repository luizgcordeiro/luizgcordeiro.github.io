\begin{theorem}
	Sejam $f$ e $g$ funções deriváveis em um ponto $a$. Então a função produto $fg$ é derivável em $a$, e
	\[(fg)'(a)=f'(a)g(a)+f(a)g'(a).\]
\end{theorem}

\begin{proof}
	Vamos utilizar a \href{../09.derivadas/con_der_em_pt_moodle.html}{definição de derivada como um limite} e as \href{../02.limites_elementares/teo_regras_lim_moodle.html}{regras de limites}.
	
	\begin{align*}
		(fg)'(a)
			&=\lim_{x\to a}\frac{(fg)(x)-(fg)(a)}{x-a}\\
			&\qquad\text{(definição de derivada)}\\
			&=\lim_{x\to a}\frac{f(x)g(x)-f(a)g(a)}{x-a}\\\\
			&\qquad\text{(definição de função produto)}\\
			&=\lim_{x\to a}\frac{f(x)g(x)-f(a)g(x)+f(a)g(x)-f(a)g(a)}{x-a}\\
			&\qquad\text{(somar zero no denominador)}\\
			&=\lim_{x\to a}\frac{\left(f(x)-f(a)\right)g(x)+f(a)\left(g(x)-g(a)\right)}{x-a}\\
			&\qquad\text{(reorganização)}\\
			&=\lim_{x\to a}\left(\frac{f(x)-f(a)}{x-a}g(x)+f(a)\frac{g(x)-g(a)}{x-a}\right)\\
			&\qquad\text{(reorganização)}\\
			&=\left(\lim_{x\to a}\frac{f(x)-f(a)}{x-a}\right)\left(\lim_{x\to a}g(x)\right)+f(a)\left(\lim_{x\to a}\frac{g(x)-g(a)}{x-a}\right)\\
			&\qquad\text{(regras de limites)}.
	\end{align*}
	
	Como a função $g$ é diferenciável no ponto $a$, \href{../09.derivadas/teo_funcoes_derivaveis_sao_continuas_moodle.html}{então $g$ é contínua em $a$}, o que significa que $\lim_{x\to a}g(x)=g(a)$. Substituindo este limite e $f'(a)$ e $g'(a)$ nos limites apropriados na última expressão obtida acima, obtemos
	\begin{align*}
		(fg)'(a)&=f'(a)g(a)+f(a)g'(a).
	\end{align*}
\end{proof}