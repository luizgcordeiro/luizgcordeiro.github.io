Os seguintes limites ''elementares'' serão utilizados constantemente. Em termos simples, sempre que tivermos uma expressão algébrica elementar $E(x)$ em uma variável $x$ -- isto é envolvendo polinômios, potências, funções trigonométricas, exponenciais, etc. -- basta trocar a variável $x$ por $a$ para obter $\lim_{x\to a}E(x)=E(a)$.

Mais geralmente, é sempre possível ``passar o limite para dentro de expressões elementares'', no sentido de que se $f$ é uma função e $E(x)$ é uma expressão elementar, então
\[\lim_{x\to a} E(f(x))=E\left(\lim_{x\to a}f(x)\right),\]
desde que o limite $\lim_{x\to a}f(x)$ exista.

Os seguintes limites são válidos, desde que o lado direito da igualdade esteja bem-definido.

\begin{itemize}
	\item \emph{Limite de funções constantes}:
	\[\lim_{x\to a}c=c.\]
	\item \emph{Limite da função identidade}:
	\[\lim_{x\to a}x=a.\]
	\item \emph{Limite de funções monomiais}:
	\[\lim_{x\to a}x^n=a^n.\]
	\item \emph{Limite de funções com expoentes racionais}:
	\[\lim_{x\to a}x^{n/m}=a^{n/m}.\]
	\item \emph{Limite de funções exponenciais}:
	\[\lim_{x\to a}c^x=c^a.\]
	\item \emph{Limite de seno}:
	\[\lim_{x\to a}\sin(x)=\sin(a).\]
	\item \emph{Limite de cosseno}:
	\[\lim_{x\to a}\cos(x)=\cos(a).\]
	\item \emph{Limite de tangente}:
	\[\lim_{x\to a}\tan(x)=\tan(a).\]
\end{itemize}
