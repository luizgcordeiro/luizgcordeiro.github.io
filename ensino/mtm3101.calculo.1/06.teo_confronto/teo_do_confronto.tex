\begin{theorem}
	Suponha que $f$, $m$ e $M$ são funções, todas definidas em intervalos ao redor de um ponto $a$, e satisfazendo
	\(m(x)\leq f(x)\leq M(x)\)]
	para todo $x$ suficientemente próximo de $a$, e que
	\[\lim_{x\to a}m(x)=\lim_{x\to a}M(x)=L.\]
	Então $\lim_{x\to a}f(x)=L$.
\end{theorem}

\begin{proof}
	Dado $\epsilon>0$, tome $\delta_m,\delta_M>0$ tais que
	\[0<|x-a|<\delta_m\quad\text{implica que}\quad |m(x)-L|<\epsilon\]
	e
	\[0<|x-a|<\delta_M\quad \text{implica que}\quad |M(x)-L|<\epsilon.\]
	
	Seja $\delta_1=\min(\delta_m,\delta_M)$. Se $0<|x-a|<\delta$, então
	\[|m(x)-L|<\epsilon\quad\text{e}\quad|M(x)-L|<\epsilon,\]
	ou equivalentemente
	\[L-\epsilon < m(x) < L+\epsilon\quad\text{e}\quad L-\epsilon < M(x) < L+\epsilon.\]
	Como $m(x)\leq f(x)\leq M(x)$ para $x$ suficientemente próximo de $a$, então existe $\delta_2>0$ tal que $0<|x-a|<\delta_2$ implica nestas desigualdades.
	
	Concluímos que se $0<|x-a|<\min(\delta_1,\delta_2)$, então
	\[L-\epsilon < m(x)\leq f(x)\leq M(x) < L+\epsilon,\]
	e portanto
	\[|f(x)-L|<\epsilon.\]
	
	Isto prova que $L=\lim_{x\to a}f(x)$.
\end{proof}