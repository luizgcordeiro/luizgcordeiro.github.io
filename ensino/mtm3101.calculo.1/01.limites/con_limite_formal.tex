\begin{definition}
	Sejam $f$ uma função e $a$ um número real.
	
	Dizemos que um número real $L$ é o \textbf{limite de $f$ em $a$} se para todo $\epsilon>0$ existir $\delta>0$ (dependendo de $\epsilon)$ tal que se $0<|x-a|<\delta$ e $x$ pertence ao domínio de $f$, então $|f(x)-L|<\epsilon$.
	
	\hrule
	
	Pelo \href{teo_unicidade_de_limite_moodle.html}{Teorema de Unicidade de Limites}, o limite de $f$ em $a$ -- caso exista -- é único, e portanto podemos denotálo por $\lim_{x\to a}f(x)=L$.
\end{definition}