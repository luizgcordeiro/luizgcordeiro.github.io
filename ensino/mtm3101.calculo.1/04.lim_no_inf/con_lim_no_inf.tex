\begin{definition}
	Seja $f$ uma função real.

	Um número real $L$ é o \textbf{limite no infinito} (\textbf{positivo}) de $f$ se para todo $\epsilon>0$ existe $A$ (dependendo de $\epsilon$) tal que se $x>A$ e $x$ pertence ao domínio de $f$, então $|f(x)-L|<\epsilon$.

	Um análogo ao \href{01.limites\teo_unicidade_de_limite_moodle.html}{Teorema de Unicidade de Limites} é válido neste caso, e o limite de $f$ no infinito -- caso exista -- é único, e portanto podemos denotá-lo por
	\[\lim_{x\to+\infty}f(x)=L.\]
	
	\hrule
	
	Similarmente, um número real $M$ é o \textbf{limite no menos infinito} (ou \textbf{infinito negativo}) de $f$ se para todo $\epsilon>0$ existe $A$ (dependendo de $\epsilon$) tal que se $x<A$ e $x$ pertence ao domínio de $f$, então $|f(x)-M|<\epsilon$.

	Um análogo ao \href{01.limites\teo_unicidade_de_limite_moodle.html}{Teorema de Unicidade de Limites} é válido neste caso, e o limite de $f$ no menos infinito -- caso exista -- é único, e portanto podemos denotá-lo por
	$\lim_{x\to-\infty}f(x)=M.$
\end{definition}