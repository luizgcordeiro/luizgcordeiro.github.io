\begin{theorem}[Regra da cadeia]
	Sejam $g$ uma função diferenciável num ponto $a$, e $f$ uma função diferenciável em $g(a)$. Então a função composta $f\circ g$ é diferenciável em $a$, e
	\[(f\circ g)'(a)=f'(g(a))g'(a).\]
\end{theorem}

Vamos prover duas demonstrações deste fato. Na primeira, vamos utilizar diretamente a definição de derivada como um limite. Na segunda, vamos utilizar funções auxiliares e propriedade de funções contínuas, o que permite simplificar vários cálculos.

\begin{proof}[Primeira demonstração]
	Para calcular $(f\circ g)'(a)$, precisamos garantir que $\left|\frac{f(g(x))-f(g(a))}{x-a}-f'(g(a))g'(a)\right|$ é pequeno quando $x$ está próximo a $a$. Para isso, vamos reescrever esta expressão em termos de $f'(g(a))$ e $g'(a)$. Podemos reescrever
	\begin{align*}
		&\frac{f(g(x)-f(g(a))}{x-a}-f'(g(a))g'(a)\\
		
			&\hspace{100pt}=\frac{f(g(x))-f(g(a))}{g(x)-g(a)}\cdot\frac{g(x)-g(a)}{x-a}\\
			&\hspace{100pt}\qquad-f'(g(a))g'(a)\\
			 & \\
			&\hspace{100pt}=\frac{f(g(x))-f(g(a))}{g(x)-g(a)}\cdot\frac{g(x)-g(a)}{x-a}\\
			&\hspace{100pt}\qquad-f'(g(a))\cdot\frac{g(x)-g(a)}{x-a}\\
			&\hspace{100pt}\qquad +f'(g(a))\cdot\frac{g(x)-g(a)}{x-a}\\
			&\hspace{100pt}\qquad-f'(g(a))g'(a)\\
			 & \\
			&\hspace{100pt}=\left(\frac{f(g(x))-f(g(a))}{g(x)-g(a)}-f'(g(a))\right)\frac{g(x)-g(a)}{x-a}\\
			&\hspace{100pt}\qquad +f'(g(a))\left(\frac{g(x)-g(a)}{x-a}-g'(a)\right),\tag{$\ast$}
	\end{align*}
	desde que $g(x)\neq g(a)$. Neste caso, os termos que aparecem na expressão final podem ser feitos tão pequenos quanto se queira, desde que tenhamos $x$ suficientemente próximo a $a$. Porém, este cálculo não pode ser feito caso existam muitos pontos $x$ próximos a $a$ tais que $g(x)=g(a)$, e temos que considerar estes pontos com mais cautela.
	
	\hrule
	
	Seja $\epsilon>0$. Como $f'(g(a))=\lim_{y\to g(a)}\frac{f(y)-f(g(a))}{y-g(a)}$, então existe $\eta>0$ tal que
	\[\left|\frac{f(y)-f(g(a))}{y-g(a)}-f'(g(a))\right|<\frac{\epsilon}{2(|g'(a)|+1)}\tag{1}\]
	sempre que $0<|y-g(a)|<\eta$.
	
	Como $g'(a)=\lim_{x\to a}\frac{g(x)-g(a)}{x-a}$ e $\eta>0$, então existe $\delta_1>0$ tal que
	\[\left|\frac{g(x)-g(a)}{x-a}-g'(a)\right|<\min\left\{\frac{\epsilon}{2(|f'(g(a))|+1)},1\right\}\tag{2}\]		
	sempre que $0<|x-a|<\delta_1$.
	
	Além disso, $g$ é contínua em $a$, portanto existe $\delta_2>0$ tal que
	\[\left|g(x)-g(a)\right|<\eta\tag{3}\]
	sempre que $|x-a|<\delta_2$.
	
	Seja $\delta=\min\left\{\delta_1,\delta_2\right\}$. Vamos provar que $\left|\frac{f(g(x)-f(g(a))}{x-a}-f'(g(a))g'(a)\right|$ sempre que $0<|x-a|<\delta$. Temos dois casos a considerar:
	
	\textbf{Caso 1}: $g(x)=g(a)$.
	\begin{quote}
		Substituindo $g(x)=g(a)$, obtemos
		\begin{align*}
			\left|\frac{f(g(x))-f(g(a))}{x-a}-f'(g(a))g'(a)\right|
				&=|f'(g(a))|\cdot|g'(a)|
		\end{align*}
		
		Mas substituindo $g(x)=g(a)$ em (2), obtemos que $|g'(a)|<\frac{\epsilon}{2(|f'(g(a))|+1)}$, logo
		\begin{align*}
			\left|\frac{f(g(x))-f(g(a))}{x-a}-f'(g(a))g'(a)\right|
				&\leq |f'(g(a))|\cdot\frac{\epsilon}{2(|f'(g(a))+1)}\\
				&<\frac{\epsilon}{2}\\
				&<\epsilon.
		\end{align*}
	\end{quote}
	
	\textbf{Caso 2}: $g(x)\neq g(a)$.
	\begin{quote}
		De (3) e da hipótese acima, obtemos $0<|g(x)-g(a)|<\eta$. De (1) com $y=g(x)$, obtemos
		\[\left|\frac{f(g(x))-f(g(a))}{g(x)-g(a)}\right|<\frac{\epsilon}{2(|g'(a)|+1)}.\]
		De (2), obtemos
		\[\left|\frac{g(x)-g(a)}{x-a}-g'(a)\right|<\min\left\{\frac{\epsilon}{2(|f'(g(a))|+1)},1\right\}\]
		Disto também segue que
		\[\left|\frac{g(x)-g(a)}{x-a}\right|< 1+|g'(a)|.\]
		Portanto, de ($\ast$) concluímos que
		\begin{align*}
		&\left|\frac{f(g(x)-f(g(a))}{x-a}-f'(g(a))g'(a)\right|\\
			&\hspace{100pt}\leq\left|\frac{f(g(x))-f(g(a))}{g(x)-g(a)}-f'(g(a))\right|\cdot\left|\frac{g(x)-g(a)}{x-a}\right|\\
			&\hspace{100pt}\qquad +\left|f'(g(a))\right|\cdot\left|\frac{g(x)-g(a)}{x-a}-g'(a)\right|\\
			& \\
			&\hspace{100pt}<\frac{\epsilon}{2(|g'(a)|+1)}\cdot(1+|g'(a)|)\\
			&\hspace{100pt}\qquad +\left|f'(g(a))\right|\cdot\frac{\epsilon}{2(|f'(g(a))|+1)}\\
			 & \\
			&\hspace{100pt}<\epsilon,
		\end{align*}
		sempre que $0<|x-a|<\delta$.
	\end{quote}
	
	Isto prova que
	\[\lim_{x\to a}\frac{f(g(x))-f(g(a))}{x-a}=f'(g(a))g'(a),\]
	ou seja, $f\circ g$ é diferenciável em $a$ e $(f\circ g)'(a)=f'(g(a))g'(a)$, como queríamos.
\end{proof}

\begin{proof}[Segunda demonstração]
	Dadas uma função $h$, um ponto $\alpha$ de seu domínio e um valor $L$, definimos uma nova função $\Phi_{h,\alpha,L}$ por
	\[\Phi_{h,\alpha,L}(x)=\begin{cases}\frac{h(x)-h(\alpha)}{x-\alpha},&\text{se }x\neq\alpha\\L,&\text{se }x=\alpha\end{cases}.\]
	
	Deste modo, as seguintes duas afirmações são equivalentes, para todas $h$, $\alpha$ e $L$:
	\begin{enumerate}
		\item $h$ é diferenciável em $\alpha$ e $h'(\alpha)=L$;
		\item $\Phi_{h,\alpha,L}$ é contínua em $\alpha$.
	\end{enumerate}
	
	Dadas duas funções $h$ e $k$, um ponto $\alpha$ e valores $L$ e $M$, vamos verificar que a seguinte igualdade é válida, para todo $x$:
	\[\Phi_{h\circ k,\alpha,LM}(x)=\Phi_{h,k(\alpha),L}(k(x))\cdot\Phi_{k,\alpha,M}(x).\tag{$\star$}\]
	De fato, temos três casos a verificar:
	
	\textbf{Caso 1}: $x=\alpha$.
	\begin{quote}
		Neste caso, também teremos $k(x)=k(\alpha)$, e assim pelas definições temos que
		\begin{align*}
			\Phi_{h\circ k,\alpha,LM}(x)&=LM\text{,}\\
			\Phi_{h,k(\alpha),L}(k(x))&=L\text{, e}\\
			\Phi_{k,\alpha,M}(x)&=M\text{,}
		\end{align*}
		e a igualdade ($\star$) é válida.
	\end{quote}
	
	\textbf{Caso 2}: $x\neq \alpha$ e $k(x)=k(\alpha)$.
	\begin{quote}
		Neste caso, $h(k(x))=h(k(\alpha))$, e assim
		\begin{align*}
			\Phi_{h\circ k,\alpha,LM}(x)
				&=\frac{h(k(x))-h(k(\alpha))}{x-\alpha}\\
				&=0
		\end{align*}
		e
		\begin{align*}
			\Phi_{k,\alpha,M}(x)
				&=\frac{k(x)-k(\alpha)}{x-\alpha}\\
				&=0.
		\end{align*}
		Portanto,
		\begin{align*}
			\Phi_{h\circ k,\alpha,LM}(x)
				&=0\\
				&=\Phi_{h,k(\alpha),L}(k(x))\cdot 0\\
				&=\Phi_{h,k(\alpha),L}(k(x))\cdot \Phi_{k,\alpha,M}(x).
		\end{align*}
	\end{quote}
	
	\textbf{Caso 3}: $x\neq \alpha$ e $k(x)\neq k(\alpha)$.
	\begin{quote}
		Neste caso, temos que
		\begin{align*}
			\Phi_{h\circ k,\alpha,LM}(x)&=\frac{h(k(x))-h(k(\alpha))}{x-\alpha}\text{,}\\
			\Phi_{h,k(\alpha),L}(k(x))&=\frac{h(k(x))-h(k(\alpha))}{k(x)-k(\alpha)}\text{, e}\\
			\Phi_{k,\alpha,M}(x)&=\frac{k(x)-k(\alpha)}{x-\alpha}.
		\end{align*}
		Assim,
		\begin{align*}
			\Phi_{h\circ k,\alpha,LM}(x)&=\frac{h(k(x))-h(k(\alpha))}{x-\alpha}\\
			&=\frac{h(k(x))-h(k(\alpha))}{k(x)-k(\alpha)}\cdot\frac{k(x)-k(\alpha)}{x-\alpha}\\
			&=\Phi_{h,k(\alpha),L}(k(x))\cdot \Phi_{k,\alpha,M}(x).
		\end{align*}
	\end{quote}
	
	A igualdade ($\star$) significa que temos a igualdade de funções (no domínio de $h\circ k$)
	\[\Phi_{h\circ k,\alpha,LM}=\left(\Phi_{h,k(\alpha),L}\circ k\right)\cdot \Phi_{k,\alpha,M}\tag{$\clubsuit$}.\]
	
	\hrule
	
	Vamos utilizar a igualdade ($\clubsuit$) com $h=f$, $k=g$, $\alpha=a$, $L=f'(g(a))$ e $M=g'(a)$, e a relação entre continuidade e diferenciabilidade satisfeita pelas funções $\Phi_{\ast,\ast,\ast}$.

	Temos que
	\[\Phi_{f\circ g,g(a),f'(g(a))g'(a)}=\left(\Phi_{f,g(a),f'(g(a))}\circ g\right)\cdot \Phi_{g,a,g'(a)}.\]
	
	Como a função $g$ é diferenciável em $a$, então $\Phi_{g,a,g'(a)}$ é contínua em $a$.
	
	Além disso, \href{../09.derivadas/teo_funcoes_derivaveis_sao_continuas_moodle.html}{$g$ também é contínua em $a$}. Como $f$ é diferenciável em $g(a)$, então $\Phi_{f,g(a),f'(g(a))}$ é contínua em $g(a)$. Como \href{../07.fun_cont/teo_comp_cont_moodle.html}{continuidade é preservada por composição}, então $\Phi_{f,g(a),f'(g(a))}\circ g$ é contínua em $a$.
	
	Como \href{../07.fun_cont/teo_prop_fun_cont_moodle.html}{continuidade é preservada por produtos}, então $\left(\Phi_{f,g(a),f'(g(a))}\circ g\right)\cdot \Phi_{g,a,g'(a)}$ é contínua em $a$, ou seja, $\Phi_{f\circ g,g(a),f'(g(a))g'(a)}$ é contínua em $a$. Isto significa que $f\circ g$ é diferenciável em $a$, e $(f\circ g)'(a)=f'(g(a))g'(a)$, exatamente como queríamos.
\end{proof}