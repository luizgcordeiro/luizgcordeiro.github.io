\begin{definition}
	Seja $f\colon X\to Y$ uma função entre conjuntos $X$ e $Y$.
	\begin{itemize}
		\item $f$ é \textbf{injetiva} ou \textbf{injetora} se $f$ leva elementos distintos de $X$ em elementos distintos de $Y$. Formalmente, isto significa que para quaisquer elementos $a,b\in X$, se $f(a)=f(b)$ então $a=b$.
		
		Equivalentemente, $f$ é injetiva se para todos $a,b\in X$ com $a\neq b$, tem-se que $f(a)\neq f(b)$.
		
		\item $f$ é \textbf{sobrejetiva} ou \textbf{sobrejetora} se para todo $y\in Y$ existir $x\in X$ tal que $f(x)=y$. Isto significa que $Y$ coincide com a imagem de $f$.
		
		\item $f$ é \textbf{bijetiva} ou \textbf{bijetora} se for simultaneamente injetiva e sobrejetiva.
	\end{itemize}
\end{definition}