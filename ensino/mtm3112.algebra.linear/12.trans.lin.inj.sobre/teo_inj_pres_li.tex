\begin{theorem}
	Se $T\colon V\to W$ é linear e injetiva, então $T$ leva conjuntos linearmente independentes de $V$ em conjuntos linearmente independentes de $W$. Mais precisamente, se $A$ é um subconjunto linearmente independente de $V$, então sua imagem direta $T(A)=\left\{T(a):a\in A\right\}$ é linearmente independente em $W$.
\end{theorem}

\begin{proof}
	Dado $A$ como no enunciado, suponha que tenhamos uma combinação linear nula com elementos de $T(A)$:
	\[\sum_{i=1}^n\lambda_i w_i=0_W,\]
	onde os $w_i$ são elementos distintos de $T(A)$ são elementos de $T(A)$ e os $\lambda_i$ são escalares. Podemos escrever cada $w_i$ como $w_i=T(a_i)$ para certos $a_i\in A$. Então a equação acima se reescreve como
	\[T\left(\sum_{i=1}^n\lambda_i a_i\right)=0_W.\]
	Como $T$ é injetiva, isto significa que $\sum_{i=1}^n\lambda_i a_i=0_V$, e além disso que os $a_i$ são dois-a-dois distintos. Como $A$ é linearmente independente, então cada coeficiente $\lambda_i$ é nulo, o que prova que $T(A)$ é linearmente independente.
\end{proof}