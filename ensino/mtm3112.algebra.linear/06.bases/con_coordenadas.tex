\begin{definition}
	Sejam $V$ um espaço vetorial e $\mathcal{B}$ uma base de $V$. Por simplicidade, vamos inicialmente considerar somente o caso em que $\mathcal{B}$ é uma base finita ordenada, i.e. $\mathcal{B}=\left\{b_1,\ldots,b_n\right\}$.
	
	Cada vetor $v\in V$ pode ser expresso de forma única como uma combinação linear de elementos de $\mathcal{B}$:
	\[
		v=\alpha_1b_1+\alpha_2b_2+\cdots+\alpha_nb_n,
		\]
	onde cada $\alpha_i$ é um escalar (dependendo de $v$). A expressão acima é chamada de a \textbf{representação de $v$ na base $\mathcal{B}$}. Cada escalar $\alpha_i$ é chamado de a $i$-ésima \textbf{coordenada} de $v$ na base $\mathcal{B}$.
	
	\hrule
	
	No caso geral, temos somente uma base $\mathcal{B}$ de $V$. Dado um vetor $v\in V$, temos novamente uma única forma de expressar $v$ como uma combinação linear de elementos de $\mathcal{B}$:
	\[
		v=\sum_{b\in\mathcal{B}}\alpha_b b,
		\]
	onde todos os $\alpha_b$ são nulos exceto por uma quantidade finita. Novamente, dizemos que cada escalar $\alpha_b$ é a \textbf{coordenada} associada a $b$ na base $\mathcal{B}$, ou a \textbf{$b$-ésima coordenada} de $v$ na base $\mathcal{B}$.
\end{definition}