\begin{definition}
	O \textbf{núcleo} ou \textbf{kernel} de uma transformação linear $T\colon V\to W$, onde $V$ e $W$ são espaços vetoriais, é definido por
	\[\ker(T)=\left\{v\in V:T(v)=0_W\right\}.\]
	
	\hrule
	
	Equivalentemente, $\ker(T)=T^{-1}(0_W)$ é a \href{http://mtm.ufsc.br/~cordeiro/ensino/mtm3112.algebra.linear/10.nucleo.imagem/rev_im_preim_moodle.html}{pré-imagem} do vetor nulo de $W$, ou ainda $\ker(T)=T^{-1}(\left\{0_W\right\})$ é a pré-imagem do subespaço zero de $W$.
	
	Como \href{http://mtm.ufsc.br/~cordeiro/ensino/mtm3112.algebra.linear/10.nucleo.imagem/teo_im_inv_subesp_trans_lin_moodle.html}{subespaços vetoriais são preservados por pré-imagens por transformaçõs lineares}, então $\ker(T)$ é um subespaço vetorial de $V$.
\end{definition}