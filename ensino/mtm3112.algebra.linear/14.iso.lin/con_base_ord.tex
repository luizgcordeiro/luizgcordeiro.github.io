\begin{definition}
	Uma \textbf{base ordenada} de um espaço vetorial $V$ de dimensão finite é uma base $\mathcal{B}=\left\{v_1,\ldots,v_n\right\}$ de $V$ na qual uma ordem foi determinada para seus elementos.
	
	Em geral, já costumamos indexar os elementos de uma base finita por números naturais. Ao escrevermos ``$\mathcal{B}=\left\{v_1,\ldots,v_n\right\}$'', já subentendemos que a ordem é dada pelos índices; i.e., que ``o primeiro elemento da base é $v_1$, o segundo é $v_2$,\ldots, o último é $v_n$''.
	
	\hrule
	
	Por vezes, bases ordenadas são denotadas de modo diferente -- por exemplo, com colchetes, como em ``$\mathcal{B}=[v_1,\ldots,v_n]$'', para reforçar a diferença com uma base ``não-ordenada''. Para os nossos fins isto não será necessário, pois a ordem de uma base sempre estará clara pelo contexto.
\end{definition}