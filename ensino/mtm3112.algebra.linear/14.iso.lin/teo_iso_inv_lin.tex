\begin{theorem}
	Seja $T\colon V\to W$ um isomorfismo linear entre espaços vetoriais $V$ e $W$. Então a função inversa $T^{-1}\colon W\to V$ também é linear.
\end{theorem}

Vamos prover duas provas distintas pra este teorema.

\begin{proof}[Prova usual]
	Sejam $w_1,w_2\in W$ e $\lambda\in\mathbb{R}$. Devemos provar que $T^{-1}(w_1+\lambda w_2)=T^{-1}(w_1)+\lambda T^{-1}(w_2)$.
	
	Sejam $v_1=T^{-1}(w_1)$ e $v_2=T^{-1}(w_2)$. Como $T$ é linear, temos que
	\begin{align*}
		T(v_1+\lambda v_2)
			&=T(v_1)+\lambda T(v_2)\\
			&=w_1+\lambda w_2.
	\end{align*}
	
	Aplicando $T^{-1}$ no primeiro e no último termos acima, obtemos a igualdade
	\[v_1+\lambda v_2=T^{-1}(w_1+\lambda w_2),\]
	ou seja,
	\[T^{-1}(w_1)+\lambda T^{-1}(w_2)=T^{-1}(w_1+\lambda w_2),\]
	como queríamos demonstrar.
\end{proof}

\begin{proof}[Prova alternativa]
	Como $T$ é injetiva, então \href{http://mtm.ufsc.br/~cordeiro/ensino/mtm3112.algebra.linear/13.trans.lin.comp/teo_inj_inv_esq_moodle.html}{$T$ admite uma inversa linear à esquerda}. Isto é, existe uma função linear $L\colon W\to V$ tal que
	\[L\circ T=\operatorname{id}_V.\]
	
	Porém, como $T$ é inversível então esta função inversa à esquerda de $T$ é na verdade a própria inversa de $T$. De fato,
	\begin{align*}
	L
		&=L\circ \operatorname{id}_W\\
		&=L\circ (T\circ T^{-1})\\
		&=(L\circ T)\circ T^{-1}\\
		&=\operatorname{id}_V\circ T^{-1}\\
		&=T^{-1}.
	\end{align*}
	
	Isto mostra que $T^{-1}=L$, que é linear por construção.
\end{prof}