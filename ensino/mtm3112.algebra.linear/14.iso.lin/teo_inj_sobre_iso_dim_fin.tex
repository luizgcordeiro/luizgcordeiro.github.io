\begin{theorem}
	Sejam $V$ e $W$ dois espaços vetoriais de dimensões finitas e iguais, $\dim(V)=\dim(W)$. Dada uma transformação linear $T\colon V\to W$, as seguintes afirmações são equivalentes:
	\begin{enumerate}
		\item $T$ é injetiva.
		\item $T$ é sobrejetiva.
		\item $T$ é um isomorfismo.
	\end{enumerate}
\end{theorem}

\begin{proof}
	Seja $n=\dim(V)=\dim(W)$. Pelo Teorema do Núcleo e da Imagem, temos que
	\[n=\dim(\ker(T))+\dim(\operatorname{Im}(T)).\tag{NI}\]
	
	Suponhamos então que $T$ é injetiva. Isto significa que $\ker(T)=\left\{0_V\right\}$, ou equivalentemente que $\dim(\ker(T))=0$. Da Equação (NI), obtemos $\dim(\operatorname{Im}(T))=n=\dim(W)$, e portanto $\operatorname{Im}(T)=W$, o que significa que $T$ é sobrejetiva.
	
	A recíproca é similar: Se supormos que $T$ é sobrejetiva, então $\dim(\operatorname{Im}(T))=n$, e da Equação (NI) segue que $\dim(\ker(T))=0$, o que significa que $T$ é injetiva.
	
	Portanto, $T$ é injetiva se, e somente se, $T$ é sobrejetiva, e neste caso $T$ acaba sendo na verdade bijetiva -- um isomorfismo linear. Reciprocamente, isomorfismos lineares são, pro definição, tanto injetivos quanto sobrejetivos.
	
	Isto mostra que qualquer uma das afirmações do teorema implica em qualquer outra, e portanto são todas equivalentes.
\end{proof}