\begin{definition}
	Seja $V$ um espaço vetorial de dimensão finita com duas bases ordenadas $\mathcal{B}=\left\{b_1,\ldots,b_n\right\}$ e $\mathcal{C}=\left\{c_1,\ldots,c_n\right\}$.
	
	A \textbf{matriz de mudança da base $\mathcal{B}$ para a base $\mathcal{C}$} é a matriz $[\operatorname{id}_V]_{\mathcal{B}}^{\mathcal{C}}$, que representa a função identidade de $V$ relativa às bases $\mathcal{B}$ e $\mathcal{C}$.
	
	\hrule
	
	Esta matriz é usada para encontrar as coordenadas de vetores de $V$ em diferentes bases. Dado um vetor $v\in V$ com vetor coordenada $[v]^{\mathcal{B}}$ relativo à base $\mathcal{B}$, temos que $[v]^\mathcal{C}=[\operatorname{id}]_{\mathcal{B}}^{\mathcal{C}}[v]^{\mathcal{B}}$.
\end{definition}

\hrule

Matrizes de mudança de bases podem ser calculadas do seguinte modo: Em geral, os espaços vetoriais que consideramos já vêm com uma base ``canônica'' (por exemplo, $\mathbb{R}^n$ com sua base padrão; O espaço $\mathbb{R}[x]$ dos polinômios reais com a base consistindo dos monômios), e as representações de vetores nessas bases canônicas são fáceis de se calcular.

Digamos então que $\mathcal{A}$ é uma base ``canônica'' ordenada de um espaço vetorial $V$ e $\mathcal{C}=\left\{c_1,\ldots,c_n\right\}$ é outra base ordenada de $V$. Considere $[c_i]^{\mathcal{A}}$, os vetores coordenadas de cada vetor de $\mathcal{C}$ na base $\mathcal{A}$ (que estamos assumindo que são fáceis de se calcular). Então por definição, temos que
\[\left[\operatorname{id}_V\right]_{\mathcal{C}}^{\mathcal{A}}=
	\begin{bmatrix}
		| & & | \\
		[c_1]^{\mathcal{A}} & \cdots & [c_n]^{\mathcal{A}} \\
		| & & |
	\end{bmatrix}\]
e portanto 
\begin{align*}
	\left[\operatorname{id}_V\right]_{\mathcal{A}}^\mathcal{C}
		&=\left(\left[\operatorname{id}_V\right]_{\mathcal{C}}^{\mathcal{A}}\right)^{-1}\\
		&=\left(\begin{bmatrix}
		| & & | \\
		[c_1]^{\mathcal{A}} & \cdots & [c_n]^{\mathcal{A}} \\
		| & & |
	\end{bmatrix}\right)^{-1}.
\end{align*}

\hrule

Mais geralmente, suponha agora que $\mathcal{B}$ e $\mathcal{C}$ sejam duas bases ordenadas de $V$, e que $\mathcal{A}$ é uma base ``canônica''. Então a matriz de mudança de base $[\operatorname{id}_V]_{\mathcal{B}}^{\mathcal{A}}$ é fácil de se calcular (suas colunas são os vetores coordenadas dos elementos de $\mathcal{B}$ na base $\mathcal{A}$), e o procedimento acima provê um método de cálculo para $[\operatorname{id}_V]_{\mathcal{A}}^\mathcal{C}$. Portanto, para calcular a matriz de mudança da base $\mathcal{B}$ para a base $\mathcal{C}$, podemos simplesmente calcular o produto destas matrizes, pois
\[\left[\operatorname{id}_V\right]_{\mathcal{B}}^{\mathcal{C}}=\left[\operatorname{id}_V\right]_{\mathcal{A}}^{\mathcal{C}}\cdot\left[\operatorname{id}_V\right]_{\mathcal{B}}^{\mathcal{A}}.\]