\begin{definition}
	Sejam $V$ um espaço vetorial e $S$ um subconjunto de $V$. O subespaço vetorial \textbf{gerado} por $S$ é definido como a intersecção $\langle S\rangle$ de todos os subespaços vetoriais de $V$ que contêm o subconjunto $S$. Formalmente,
	\[\langle S\rangle=\bigcap\left\{A:A\text{ é subespaço de }V\text{ e }S\subseteq A\right\}.\]
	Como \href{http://mtm.ufsc.br/~cordeiro/ensino/mtm3112.algebra.linear/02.subesp/teo_interseccao_subespacos_moodle.html}{subespaços vetoriais são preservados por intersecções}, $\langle S\rangle$ também é um subespaço vetorial de $V$. Em termos mais simples, $\langle S\rangle$ é o ``menor'' subespaço de $V$ que contém $S$.
\end{definition}