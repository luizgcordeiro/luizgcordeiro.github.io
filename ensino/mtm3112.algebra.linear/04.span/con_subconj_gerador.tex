\begin{definition}
	Um subconjunto $S$ de um espaço vetorial $V$ é dito ser \textbf{gerador} se $\langle S\rangle$, o \href{http://mtm.ufsc.br/~cordeiro/ensino/mtm3112.algebra.linear/04.span/con_subesp_ger_moodle.html}{subespaço gerado} por $S$, for igual a $V$.
	
	Equivalentemente, como \href{http://mtm.ufsc.br/~cordeiro/ensino/mtm3112.algebra.linear/04.span/teo_span_subesp_ger_moodle.html}{$\langle S\rangle=\operatorname{span}(S)$}, o subconjunto $S$ é gerador de $V$ se todo elemento não-nulo de $V$ pode ser expresso como uma \href{http://mtm.ufsc.br/~cordeiro/ensino/mtm3112.algebra.linear/04.span/con_comb_lin_moodle.html}{combinação linear} de elementos de $S$.
\end{definition}