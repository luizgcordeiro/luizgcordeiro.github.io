\begin{definition}
	Sejam $V$ e $W$ espaços vetoriais. uma \textbf{transformação linear} (também chamada de \textbf{função linear} ou \textbf{aplicação linear}) de $V$ em $W$ é uma função $T\colon V\to W$ que satisfaz às seguintes propriedades:
	\begin{enumerate}
		\item $T(v+w)=T(v)+T(w)$ para todos $v,w\in V$.
		\item $T(\lambda v)=\lambda T(v)$ para todos $v\in V$ e $\lambda\in\mathbb{R}$.
	\end{enumerate}
	
	\hrule

	Equivalentemente, $T\colon V\to W$ é linear se, e somente se
	\[T(v+\lambda w)=T(v)+\lambda T(w)\]
	para todos $v,w\in V$ e todo $\lambda\in\mathbb{R}$. (Esta condição utiliza ambas as operações de espaço vetorial ao mesmo tempo, e é muitas vezes mais rápido verficar somente esta condição ao invés das duas condições originais.) 
\end{definition}