\begin{theorem}
	Seja $T\colon V\to W$ uma transformação linear entre dois espaços vetoriais. Então
	\begin{enumerate}
		\item $T(0_V)=0_W$.
		\item $T(-v)=-T(v)$ para todo $v\in V$.
	\end{enumerate}
\end{theorem}

\begin{proof}[Primeira prova]
	Basta utilizar a propriedade de que $T$ preserva produtos por escalares e as Regras de Sinal. Calculamos
	\[T(0_V)=T(0\cdot 0_V)=0\cdot T(0_V)=0_W.\]
	Similarmente,
	\[T(-v)=T((-1)\cdot v)=(-1)\cdot T(v)=-T(v)\]
	para qualquer $v\in V$.
\end{proof}

\begin{proof}[Segunda prova]
	Podemos também utilizar somente a propriedade de que $T$ preserva somas e as Leis de Cancelamento. Por um lado
	\[T(0_V)=T(0_V+0_V)=T(0_V)+T(0_V).\]
	Somando $-T(0_V)$ em ambos os lados da igualdade acima, concluímos que $0_W=T(0_V)$.
	
	Similarmente, dado $v\in V$, temos
	\[T(v)+T(-v)=T(v+(-v))=T(0_V)=0_W,\]
	onde a última igualdade foi verificada acima. Assim, $T(-v)$ somado com $T(v)$ resulta no vetor nulo. Esta é precisamente a propriedade que determina o oposto de $T(v)$, Portanto, $T(-v)=-T(v)$.
\end{proof}