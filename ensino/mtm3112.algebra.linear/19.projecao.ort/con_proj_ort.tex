\begin{definition}
	Sejam $V$ um espaço com produto interno e $W$ um subespaço de $V$. Uma transformação linear $P\colon V\to W$ é dita ser uma projeção ortogonal de $V$ sobre $W$ se satisfizer às seguintes propriedades:
	\begin{enumerate}
		\item Para todo $w\in W$, $P(w)=w$.
		\item Para todo $v\in V$, $v-P(v)$ é ortogonal a $W$.
	\end{enumerate}
	Caso exista, a projeção ortogonal de $V$ sobre $W$ é única, e a denotaremos por $\operatorname{proj}_W$.
	
	\hrule
	
	Caso $W$ tenha dimensão finita, a projeção ortogonal de $V$ sobre $W$ sempre existe, e pode ser calculada explicitamente da seguinte forma: Considere uma base ortonormal $\left\{e_1,\ldots,e_n\right\}$ de $W$. Dado $v\in V$, temos que
	\[\operatorname{proj}_W(v)=\sum_{i=1}^n\langle v,e_i\rangle e_i.\]
\end{definition}