\begin{theorem}
	Sejam $V$ um espaço vetorial com produto interno e $W$ um subespaço vetorial de $V$ de dimensão finita. Seja $\left\{w_1,\ldots,w_k\right\}$ uma base ortonormal de $W$.
	
	Então existe uma projeção ortogonal de $V$ sobre $W$, que é dada pela transformação linear $P\colon V\to W$ definida por
	\[P(v)=\sum_{i=1}^k\langle v,w_i\rangle w_i\]
	para todo $v\in V$.
\end{theorem}

\begin{proof}
	Como $\left\{w_1,\ldots,w_k\right\}$ é uma base ortonormal de $W$, então cada vetor $w\in W$ é representado nesta base por
	\[w=\sum_{i=1}^k\langle w,w_i\rangle w_i,\]
	o que significa que $P(w)=w$ para todo $w\in W$.
	
	Para concluir que $P$ é, de fato, uma projeção ortogonal de $V$ sobre $W$, basta verificar que $v-P(v)$ é ortogonal a $W$ para todo $v\in V$. Para isto, é suficiente verificar que $v-P(v)$ é ortogonal a $w_j$ para cada $j$. De fato, temos que
	\begin{align*}
	\langle v-P(v),w_j\rangle
		&=\langle v,w_j\rangle-\langle P(v),w_j\rangle\\
		&=\langle v,w_j\rangle-\left\langle\sum_{i=1}^n(\langle v,w_i\rangle) w_i,w_j\right\rangle\\
		&=\langle v,w_j\rangle-\sum_{i=1}^n(\langle v,w_i\rangle)\langle w_i,w_j\rangle
	\end{align*}
	
	Como $\left\{w_1,\ldots,w_n\right\}$ é ortonormal, então $\langle w_i,w_j\rangle$ é igual a $1$ caso $i=j$, e igual a $0$ caso $i\neq j$. Assim, obtemos
	\begin{align*}
	\langle v-P(v),w_j\rangle
		&=\langle v,w_j\rangle-\langle v,w_j\rangle\\
		&=0,
	\end{align*}
	exatamente como queríamos provar.
\end{proof}