\begin{definition}
	Sejam $V$ um espaço vetorial com produto interno, $W$ um subespaço de $V$ e $W^\perp$ seu complemento ortogonal.

	A \textbf{reflexão ortogonal} de $V$ sobre $W$ é a transformação linear $\operatorname{ref}_W\colon V\to V$ definida por
	\[\operatorname{ref}_W(v)= 2\operatorname{proj}_W(v)-v,\]
	onde $\operatorname{proj}_W$ é a projeção ortogonal de $V$ em $W$.

	\hrule

	Podemos descrever a reflexão ortogonal de $V$ sobre $W$ alternativamente do seguinte modo: Definimos
	\begin{itemize}
		\item $\operatorname{ref}_W(w)=w$ se $w\in W$;
		\item $\operatorname{ref}_W(u)=-u$ se $u\in W^\perp$.
	\end{itemize}
	
	Como $V$ se decompõe como a soma direta $V=W\oplus W^\perp$, os itens acima definem unicamente uma aplicação linear $\operatorname{ref}_W$ de $V$ em $V$.
\end{definition}