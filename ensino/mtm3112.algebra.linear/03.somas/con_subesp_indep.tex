\begin{definition}
	Dois subespaços vetoriais $A$ e $B$ de um espaço vetorial $V$ são ditos \textbf{independentes} se $A\cap B=\left\{0_V\right\}$.
	
	\hrule
	
	Mais geralmente, dizemos que uma família $\left\{A_1,A_2,\ldots,A_n\right\}$ de subespaços de $V$ é \textbf{independente} se para todo $i$ vale que
	\[A_i\cap \left(\sum_{j\neq i}A_j\right)=\left\{0_V\right\},\]
	onde $\sum_{j\neq i}A_j$ denota a \href{http://mtm.ufsc.br/~cordeiro/ensino/mtm3112.algebra.linear/03.somas/con_soma_de_subesp_moodle.html}{soma} dos subespaços $A_1,\ldots,A_n$ \emph{exceto} $A_i$.
\end{definition}

\emph{Observações}:
\begin{itemize}
\item Dois subespaços $A,B$ são independentes se, e somente se, a coleção $\left\{A,B\right\}$ é independente.
\item A terminologia de subespaços ``independentes'' é amplamente aceita, mas nem sempre é utilizada.
\item A noção de independência de subespaços pode ser estendida para famílias infinitas de subespaços: Se $\mathfrak{A}$ é uma família qualquer (possivelmente infinita) de subespaços vetoriais de um espaço vetorial $V$, dizemos que $\mathfrak{A}$ é \textbf{independente} se para todo $A\in\mathfrak{A}$ vale que $A\cap\left(\sum\left(\mathfrak{A}\setminus\left\{A\right\}\right)\right)=\left\{0_V\right\}$.
\end{itemize}