\begin{definition}
	Sejam $V$ um espaço vetorial e $A$ e $B$ dois subespaços vetoriais de $V$.
	
	A \textbf{soma} de $A$ com $B$ é o conjunto consistindo de todas as somas de elementos de $A$ com elementos de $B$. Formalmente,
	\[A+B=\left\{a+b:a\in A,\ b\in B\right\}.\]
	
	\hrule
	
	Mais geralmente, a \textbf{soma} de uma coleção $\left\{A_1,\ldots,A_n\right\}$ de subespaços vetoriais de $V$ é
	\begin{align*}
		\sum_{i=1}^n A_i&=A_1+\cdots+A_n\\
		&=\left\{a_1+\cdots+a_n:a_i\in A_i\text{ para cada }i\right\}.
		\end{align*}
	\hrule
	
	A soma de subespaços vetoriais de $V$ também é um subespaço vetorial, conforme é demonstrado no \href{http://mtm.ufsc.br/~cordeiro/ensino/mtm3112.algebra.linear/03.somas/teo_soma_e_subespaco_moodle.html}{teorema apropriado}.
\end{definition}

\emph{Observação}:
\begin{itemize}
	\item A noção de soma de subespaços vetoriais pode ser estendida para famílias infinitas: Se $\mathfrak{A}$ é uma família qualquer (possivelmente infinita) de subespaços vetoriais de um espaço vetorial $V$, a \textbf{soma} de $\mathfrak{A}$ é o conjunto $\sum\mathfrak{A}$ consistindo de todas as somas de vetores $a_1+a_2+\cdots+a_n$, onde cada vetor $a_i$ pertence a algum subespaço que pertence a $\mathfrak{A}$.
	\item É comum ``estender'' todas as operações de um espaço vetorial para subconjuntos arbitrários de $V$. Ou seja, se $X$ e $Y$ são subconjuntos (não necessariamente subespaços) de $V$, $\lambda\in\mathbb{R}$ e $v\in V$, denota-se
\[X+Y=\left\{x+y:x\in X,\ y\in Y\right\};\]
\[\lambda X=\left\{\lambda x:x\in X\right\};\]
\[X-v=\left\{x-v:x\in X\right\};\]
e assim por diante. Porém, vamos evitar utilizar notações muito rebuscadas.
\end{itemize}