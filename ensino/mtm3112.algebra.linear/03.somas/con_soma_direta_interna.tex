\begin{definition}
	Sejam $A$ e $B$ dois \href{http://mtm.ufsc.br/~cordeiro/ensino/mtm3112.algebra.linear/03.somas/con_subesp_indep_moodle.html}{subespaços independentes} de um espaço vetorial $V$ (i.e., $A\cap B=\left\{0\right\}$). Neste caso, dizemos que a \href{http://mtm.ufsc.br/~cordeiro/ensino/mtm3112.algebra.linear/03.somas/con_soma_de_subesp_moodle.html}{soma} $A+B$ é uma \textbf{soma direta} (\textbf{interna}), e a denotamos por $A\oplus B$.
	
	\hrule
	
	Mais geralmente, se $\left\{A_1,A_2,\ldots,A_n\right\}$ é uma família \href{http://mtm.ufsc.br/~cordeiro/ensino/mtm3112.algebra.linear/03.somas/con_subesp_indep_moodle.html}{subespaços independentes}{independente} de subespaços vetoriais de $V$, dizemos que a \href{http://mtm.ufsc.br/~cordeiro/ensino/mtm3112.algebra.linear/03.somas/con_soma_de_subesp_moodle.html}{soma} $A_1+\cdots+A_n$ é uma \textbf{soma direta} (\textbf{interna}), e a denotamos por
	\[\bigoplus_{i=1}^n A_i=A_1\oplus A_2\oplus\cdots\oplus A_n.\]
\end{definition}

\emph{Observação}: Novamente, é possível considerar somas diretas de famílias infinitas de subespaços.