\begin{definition}
	Dada uma matriz $A\in M_{m\times n}(\mathbb{R})$, o \textbf{espaço nulo} ou \textbf{kernel} de $A$ é o subespaço vetorial de $\mathbb{R}^n$ definido por
	\[\operatorname{Nul}(A)=\left\{(x_1,\ldots,x_n)\in\mathbb{R}^n:A\begin{bmatrix}x_1\\\vdots\\x_n\end{bmatrix}=0_{m\times 1}\right\}.\]
	Em termos mais familiares, $\operatorname{Nul}(A)$ é o conjunto solução do sistema linear homogêneo ``$Ax=0$''.

	Se identificarmos $\mathbb{R}^n$ com os espaço $M_{n\times 1}(\mathbb{R})$ das matrizes coluna $n\times 1$, o espaço nulo de $A$ também pode ser identificado como
	\[\operatorname{Nul}(A)=\left\{x\in M_{n\times 1}(\mathbb{R}):Ax=0_{m\times 1}\right\}.\]

	\hrule

	\textit{Importante}: Na definição de $\operatorname{Nul}(A)$, tomamos os vetores que quando multiplicados à esquerda por $A$ criam o vetor nulo. Caso invertêssemos a ordem do produto, obteríamos o \textbf{espaço conulo} ou \textbf{cokernel} de $A$:
	\[\operatorname{CoNul}(A)=\left\{x\in\mathbb{R}^m:xA=0_{1\times n}\right\}.\]

	Note que $\operatorname{CoNul}(A)=\operatorname{Nul}(A^T)$. Os espaços nulo e conulo de uma matriz são geralmente diferentes.
	
	\hrule
	
	\textit{Observação}: O espaço nulo de uma matriz $A$ também é denotado por $\operatorname{ker}(A)$. Por enquanto, vamos evitar o uso desta notação para evitar confusões com o kernel de transformações lineares, que serão estudadas a seguir.
\end{definition}