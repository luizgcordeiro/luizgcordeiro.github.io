\begin{theorem}
	Sejam $A$ uma matrix $m\times n$ e $R$ a forma reduzida por linhas de $A$.
	
	Então as linhas que contêm os pivôs de $R$ formam uma base para o espaço linha de $A$.
\end{theorem}
\begin{proof}
	Note que as linhas que contêm os pivôs de $R$ formam uma base para o espaço linha de $R$. De fato, $R$ estar em forma reduzida por linhas significa que
	\[R=\begin{bmatrix}-&v_1&-\\&\vdots&\\-&v_k&-\\-&0&-\\&\vdots&\\-&0&-\end{bmatrix}\]
	onde os $v_i=(v_{i1},\ldots,v_{in})$ são vetores (linhas) tais que, se o pivô de $v_i$ se encontra em uma entrada $p$, então as $p$-ésimas entradas de $v_{i+1}$, $v_{i+2}$, \ldots, $v_k$ são todas iguais a zero.
	
	As linhas $v_1,\ldots,v_k$, que contêm os pivôs de $R$, claramente geram o espaço linha de $R$. Quanto a elas serem linearmente independentes, suponha que temos uma combinação linear nula:
	\[\lambda_1 v_1+\cdots+\lambda_k v_k=0.\]
	Vamos olhar para a entrada do pivô de $v_1$: Pela propriedade descrita acima, as entradas correspondentes de $v_2,\ldots,v_k$ são todas nulas. Portanto, se o pivô está na $p$-ésima entrada de $v_1$, obtemos
	\[\lambda_1 v_{1p}=0,\]
	e portanto $\lambda_1=0$. A combinação linear nula se transforma em
	\[\lambda_2 v_2+\cdots+\lambda_k v_k=0.\]
	Repetimos o mesmo argumento para $v_2$, olhando para a entrada que contém seu pivô, e depois para $v_3$, $v_4$, \ldots, $v_k$, até concluir que $\lambda_1=\lambda_2=\cdots=\lambda_k=0$.
	
	Isto mostra que $v_1,\ldots,v_k$ são linearmente independentes e geradores para o espaço linha de $R$, e portanto geram uma base para este espaço.
	
	Para concluir o teorema, basta provar que espaço linha de $R$ e o espaço linha de $A$ coincidem. Para isto, vamos demonstrar que operações elementares em linhas não alteram espaços linha.
	
	Suponha que uma matriz $C$ é formada a partir de uma matriz $B$ por uma única operação elementar de linhas. Sejam $b_1,\ldots,b_m$ as linhas de $B$. Existem três operações que podem ter sido feitas:
	\begin{itemize}
		\item Troca de linhas.

		Neste caso, $C$ é obtida de $B$ trocando duas linhas de posição. Neste caso, as linhas de $B$ e $C$ são exatamente as mesmas, a menos de uma reordenação, e é claro que os espaços linhas de $B$ e $C$ são iguais, pois têm os mesmos geradores.
		
		\item Multiplicar uma linha por um escalar não-nulos.
		
		Neste caso, uma das linhas de $C$ é obtida ao se multiplicar uma linha de $B$ por um escalar não-nulo. Digamos que uma linha $b_i$ foi multiplicada por um escalar $\lambda\neq 0$. Neste caso, as linhas de $C$ são $b_1,\ldots,b_{i-1},\lambda b_i,b_{i+1},\ldots,b_m$, enquanto as linhas de $B$ são $b_1,\ldots,b_m$. Claramente, estes conjuntos de vetores geram o mesmo subespaço. Portanto $\operatorname{Lin}(C)=\operatorname{Lin}(B)$.
		
		\item Adicionar um múltiplo de uma linha a outra.
		
		Isto significa que as linhas de $C$ são as mesmas que as de $B$, exceto que para algum $i$, a linha $b_i$ é trocada por um vetor da forma $b_i+\lambda b_j$ para algum $j\neq i$. De qualquer modo, ambas as linhas de $C$ e de $B$ geram o mesmo subespaço, como é fácil de verificar.
	\end{itemize}
	
	Em todos os casos, $\operatorname{Lin}(C)=\operatorname{Lin}(B)$.
	
	Voltando à prova do teorema, a matriz $R$ pode ser obtida a partir de $A$ a por uma quantidade finita de operações elementares em linhas. Como cada uma dessas operações elementares preserva espaços linha, então o espaço linha de $R$ coincide com o espaço linha de $A$.
\end{proof}