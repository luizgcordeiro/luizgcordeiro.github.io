\begin{theorem}
	Sejam $A$ uma matrix $m\times n$ e $R$ a forma reduzida por linhas de $A$.
	
	Então as colunas de $A$ que correspondem às colunas que contêm os pivôs de $R$ formam uma base para o espaço coluna de $A$.
\end{theorem}
\begin{proof}
	A demonstração deste teorema é extremamente simplificada utilizando alguns resultados simples acerca de transformações lineares, que serão tratadas mais à frente no curso.
	
	Note que as colunas de $R$ que contêm seus pivôs são elementos da base canônica de $\mathbb{R}^m$, e portanto são linearmente independentes. Além disso, as únicas entradas não-nulas das outras colunas de $R$ estão em alguma linha que contém um pivô, e portanto as colunas que contêm pivôs geram o espaço coluna de $R$.
	
	Como $R$ é obtida a partir de $A$ por operações elementares sobre linhas, então existe uma matriz inversível $I$ tal que $IR=A$.
	
	Vamos reescrever as matrizes $R$ e $A$ em função de suas colunas:
	\[R=\begin{bmatrix}|&&|\\
	r_1&\cdots&r_n\\
	|&&|\end{bmatrix}\]
	e
	\[A=\begin{bmatrix}|&&|\\
	a_1&\cdots&a_n\\
	|&&|\end{bmatrix}.\]
	Então
	\begin{align*}
	\begin{bmatrix}|&&|\\
	a_1&\cdots&a_n\\
	|&&|\end{bmatrix}&=A\\
	&=IR\\
	&=I\begin{bmatrix}|&&|\\
	r_1&\cdots&r_n\\
	|&&|\end{bmatrix}\\
	&=\begin{bmatrix}|&&|\\
	Ir_1&\cdots&Ir_n\\
	|&&|\end{bmatrix}.
	\end{align*}
	Isto significa que $a_i=Ir_i$ para todo $i$, e que $\operatorname{Col}(A)=\langle Ir_1,\ldots,Ir_n\rangle=I(\operatorname{col}(R))$.
	
	Consideremos agora as colunas $r_{i_1},\ldots,r_{i_k}$ que contêm pivôs de $R$. Como já vimos, $r_{i_1},\ldots,r_{i_k}$ formam uma base para $\operatorname{Col}(R)$. Portanto, $a_{i_1}=Ir_{i_1}$,\ldots $a_{i_k}=Ir_{i_k}$ formam uma base para $\operatorname{Col}(A)$, visto que $I$ é inversível. Estas colunas de $A$ são exatamente as colunas correspondentes às colunas que contêm pivôs de $R$.
\end{proof}