\begin{definition}
	Dada uma matriz $A\in M_{m\times n}(\mathbb{R})$, as linhas de $A$ podem ser vistas como elementos de $\mathbb{R}^n$. O \textbf{espaço linha} de $A$ é o subespaço de $\mathbb{R}^n$ gerado pelas linhas de $A$. Vamos denotá-lo por $\operatorname{Lin}(A)$.

	Mais precisamente, se a matriz $A$ é dada por
	\[A=\begin{bmatrix}
		a_{11}
			&a_{12}
				&\cdots
					&a_{1n}\\
		a_{21}
			&a_{22}
				&\cdots
					&a_{2n}\\
		\vdots
			&\vdots
				&\ddots
					&\vdots\\
		a_{m1}
			&a_{m2}
				&\cdots
					&a_{mn}
		\end{bmatrix},\]
	então consideramos cada linha $l_i=(a_{i1},a_{i2},\ldots,a_{in})$, e definimos $\operatorname{Lin}(A)=\langle l_1,\ldots,l_m\rangle$.
	
	\hrule
	
	\textit{Observação}: O espaço linha de $A$ coincide com o \href{http://mtm.ufsc.br/~cordeiro/ensino/mtm3112.algebra.linear/08.espacos.de.matrizes/con_esp_coluna_moodle.html}{espaço coluna} de $A^T$, isto é, $\operatorname{Lin}(A)=\operatorname{Col}(A^T)$.
\end{definition}