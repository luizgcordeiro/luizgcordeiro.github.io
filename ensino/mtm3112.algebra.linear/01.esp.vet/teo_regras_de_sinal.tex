\begin{theorem}[Regras de Sinal]
	Seja $V$ um espaço vetorial. Então as seguintes igualdades são válidas para quaisquer escalares $\alpha$ e $\beta$ e quaisquer vetores $v$ e $w$:
	\begin{enumerate}
		\item $\alpha 0_V=0_V$.
		\item $0v=0_V$.
		\item $-(-v)=v$.
		\item $\alpha(-v)=(-\alpha)v=-(\alpha v)$.
		\item $\alpha(v-w)=(\alpha v)-(\alpha w)$.
		\item $(\alpha-\beta)v=(\alpha v)-(\beta v)$.
	\end{enumerate}
\end{theorem}
\begin{proof}
	\begin{enumerate}
		\item Consideremos $0_V$ o vetor nulo. Por um lado, a definição de vetor nulo e a propriedade distributiva implicam que
		\[\alpha 0_V=\alpha(0_V+0_V)=(\alpha 0_V)+(\alpha 0_V).\]
		Por outro lado, a definição de vetor nulo implica que
		\[\alpha 0_V=(\alpha 0_V)+0_V.\]
		Os lados esquerdos das igualdades acima são iguais, logo os lados direitos também são iguais. Isto é,
		\[(\alpha 0_V)+(\alpha 0_V)=(\alpha 0_V)+0_V.\]
		Pela Lei do Cancelamento, $\alpha 0_V=0_V$.
		\item Este item é provado de forma similar ao anterior. Sucintamente, as propriedades do escalar $0$ e a distributividade implicam que
		\[0v+0v=(0+0)v=0v=0v+0.\]
		Pela Lei do Cancelamento, $0v=0_V$.
		\item Como $-v$ é o oposto de $v$, então
		\[v+(-v)=(-v)+v=0_V.\]
		Agora lembre-se da definição do oposto de $-v$: é o único vetor que somado a $-v$ resulta em $0_V$. Ora, as igualdades acima significam precisamente que $v$ satisfaz a esta definição. Portanto $v$ é o oposto de $-v$: $v=-(-v)$.
		\item Vamos utilizar o primeiro item. Por distributividade,
		\begin{align*}
			(\alpha v)+(\alpha(-v))&=\alpha(v-v)\tag{distributividade}\\
			&=\alpha 0_V\tag{oposto}\\
			&=0_V.\tag{item 1}
		\end{align*}
		Portanto $\alpha(-v)$ satisfaz à propriedade que define o oposto de $\alpha v$. Isto significa que $\alpha(-v)=-(\alpha v)$. A igualdade $(-\alpha)v=-(\alpha v)$ é provada de modo totalmente análogo.
		\item Temos que
		\begin{align*}
			\alpha(v-w)&=(\alpha v)+(\alpha(-w))\tag{distributividade}\\
			&=\alpha v-\alpha w.\tag{item 4}
		\end{align*}
		\item Este item é provado de modo inteiramente análogo ao item anterior.
	\end{enumerate}
\end{proof}