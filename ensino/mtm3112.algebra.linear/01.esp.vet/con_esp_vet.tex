\begin{definition}
	Um \textbf{espaço vetorial} consiste de um conjunto não vazio $V$, cujos elementos são chamados de \textbf{vetores}, munido de duas operações:
	\begin{itemize}
		\item A \textbf{soma} ou \textbf{adição} de vetores, que a cada par de vetores $u$, $v$ faz corresponder um novo vetor $u+v$, chamado de \textbf{soma} de $u$ e $v$.
		\item A \textbf{multiplicação} ou \textbf{produto por escalar}, que a cada vetor $v$ e a cada número real $\alpha$ associa um novo vetor $\alpha v$, chamado de produto de $\alpha$ por $v$.
	\end{itemize}
	
	Estas operações também devem satisfazer às seguintes condições abaixo (chamadas de axiomas de espaços vetoriais): Para quaisquer vetores $u$, $v$, $w$ e quaisquer números reais $\alpha$ e $\beta$:
	\begin{enumerate}
		\item \uline{Comutatividade}: $u+v=v+u$;
		\item \uline{Associatividade da soma}: $u+(v+w)=(u+v)+w$;
		\item \uline{Associatividade da multiplicação}: $\alpha(\beta v)=(\alpha\beta)v$;
		\item \uline{Existência de vetor nulo}: Existe um vetor $0_V$ tal que $v+0_V=0_V+v=v$ para todo $v\in V$;
		\item \uline{Existência de inverso aditivo}: Para todo vetor $v$, existe um vetor $−v$, tal que $v+(−v)=(-v)+v=0_V$;
		\item \uline{Distributividade à esquerda}: $\alpha(u+v)=\alpha u+\alpha v$;
		\item \uline{Distributividade à direita}: $(\alpha+\beta)v=\alpha v+\beta v$;
		\item \uline{Multiplicação por $1$}: $1v=v$.
	\end{enumerate}
	
	O vetor $0_V$ no item 4 é chamado de \textbf{vetor nulo}, \textbf{origem} ou \textbf{zero} do espaço vetorial $V$, e o vetor $-v$ do item 5 é chamado de \textbf{oposto} de $v$. Conforme os teoremas de \href{http://mtm.ufsc.br/~cordeiro/ensino/mtm3112.algebra.linear/01.esp.vet/teo_unicidade_vetor_nulo_moodle.html}{Unicidade do vetor nulo} e de \href{http://mtm.ufsc.br/~cordeiro/ensino/mtm3112.algebra.linear/01.esp.vet/teo_unicidade_vetor_oposto_moodle.html}{Unicidade do vetor oposto} demonstram, estas propriedades determinam unicamente os vetores nulo e opostos.
\end{definition}