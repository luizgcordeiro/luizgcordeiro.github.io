Seja $V$ um espaço vetorial. Lembre-se que um \emph{vetor nulo} de $V$ é um vetor $0_V$ tal que $0_V+v=v+0_V=v$ para todo $v\in V$. Um dos axiomas de espaço vetorial nos diz que existe pelo menos um vetor nulo em $V$.

\begin{theorem}
	O vetor nulo de $V$ é único.
\end{theorem}
\begin{proof}
	Suponha que $0_V$ e $0_V'$ sejam vetores nulos de $V$. Devemos mostrar que $0_V=0_V'$. Por um lado, $0_V+0_V'=0_V'$, pois $0_V$ é um vetor nulo. Por outro lado, $0_V+0_V'=0_V$, pois $0_V'$ também é um vetor nulo. Assim, concluímos que
	\begin{align*}
		0_V
			&=0_V+0_V'\\
			&=0_V',
	\end{align*}
	conforme queríamos demonstrar. 
\end{proof}