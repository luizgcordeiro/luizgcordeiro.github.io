\begin{definition}
	Sejam $V$ e $W$ dois espaços vetoriais de dimensão finita. Considere bases ordenadas $\mathcal{B}=\left\{v_1,\ldots,v_n\right\}$ de $V$ e $\mathcal{C}=\left\{w_1,\ldots,w_m\right\}$ de $W$.

	Dada uma transformação linear $T\colon V\to W$, escrevemos cada $T(v_i)$ como uma combinação linear dos elementos de $\mathcal{C}$:

	\[\left\{\begin{array}{c c c c c c c c c}
			T(v_1)
				&=&t_{11}w_1
					&+&t_{21}w_2
						&+&\cdots
							&+&t_{m1}w_m\\
			T(v_2)
				&=&t_{12}w_1
					&+&t_{22}w_2
						&+&\cdots
							&+&t_{m2}w_m\\
				&\vdots&
					& &
						& &
							& &\\
			T(v_j)
				&=&t_{1j}w_1
					&+&t_{2j}w_2
						&+&\cdots
							&+&t_{mj}w_m\\
				&\vdots&
					& &
						& &
							& &\\
			T(v_n)
				&=&t_{1n}w_1
					&+&t_{2n}w_2
						&+&\cdots
							&+&t_{mn}w_m
		\end{array}\right.
		.\]

	A matriz de $T$ em relação às bases $\mathcal{B}$ e $\mathcal{C}$ é definida como a transposta da matriz dos coeficientes $t_{ij}$ acima, e é denotada por $[T]_{\mathcal{B}}^{\mathcal{C}}$:
	\[[T]_{\mathcal{B}}^{\mathcal{C}}=
		\begin{bmatrix}
			t_{11}
				&t_{12}
					&\cdots
						&t_{1n}\\
			t_{21}
				&t_{22}
					&\cdots
						&t_{2n}\\
			\vdots
				&\vdots
					&\ddots
						&\vdots\\
			t_{m1}
				&t_{m2}
					&\cdots
						&t_{mn}
		\end{bmatrix}.\]

	\hrule
	
	Equivalentemente:
	\begin{itemize}
		\item A $(i,j)$-ésima entrada de $[T]_{\mathcal{B}}^{\mathcal{C}}$ contém a $i$-ésima coordenada em $\mathcal{C}$ da "$j$-ésima imagem" $T(v_j)$.
		\item Se escrevermos $[T]_{\mathcal{B}}^{\mathcal{C}}$ em função de suas colunas temos que
		\[[T]_{\mathcal{B}}^{\mathcal{C}}=
			\begin{bmatrix}|&&|\\
			[T(v_1)]^{\mathcal{C}}&\cdots&[T(v_n)]^{\mathcal{C}}\\
			|&&|\end{bmatrix},
			\]
			onde cada vetor coluna ``$[T(v_j)]^{\mathcal{C}}$'' é o vetor coluna das coordenadas de $T(v_j)$ na base $\mathcal{C}$.
	\end{itemize}
\end{definition}