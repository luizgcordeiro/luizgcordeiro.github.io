Recorde que, se $T\colon V\to W$ é uma transformação linear, denotamos por $[T]_{\mathcal{B}}^{\mathcal{C}}$ a \href{http://mtm.ufsc.br/~cordeiro/ensino/mtm3112.algebra.linear/15.matrizes.trans.lin/con_rep_matr_trans_lin_moodle.html}{matriz de $T$} relativa às \href{../14.iso.lin/con_base_ord_moodle.html}{bases ordenadas} $\mathcal{B}$ e $\mathcal{C}$ de $V$ e $W$, respectivamente, e por $[v]^{\mathcal{B}}$ o \href{../14.iso.lin/con_vetor_coordenada_moodle.html}{vetor coordenada} de um elemento de $V$ com respeito à base ordenada $\mathcal{B}$ (e similarmente $[w]^{\mathcal{C}}$ para $w\in W$).

\begin{theorem}
	Sejam $V$ e $W$ espaços vetoriais de dimensão finita, com $\dim(V)=n$ e $\dim(W)=m$. Sejam $\mathcal{B}=\left\{v_1,\ldots,v_n\right\}$ e $\mathcal{C}=\left\{w_1,\ldots,w_m\right\}$ bases ordenadas de $V$ e de $W$, respectivamente. 
	\begin{enumerate}
		\item Para toda transformação linear $T\colon V\to W$ e para todo vetor $v\in V$, vale que
		\[[T(v)]^{\mathcal{C}}=[T]_{\mathcal{B}}^{\mathcal{C}}[v]^{\mathcal{B}},\]
		em que o produto na direita é o produto de matrizes.
		\item Para toda matriz $A\in M_{m\times n}(\mathbb{R})$, existe uma única transformação linear $T\colon V\to W$ tal que $[T]_{\mathcal{B}}^{\mathcal{C}}=A$.
		\item Seja $\operatorname{L}(V,W)$ o \href{../15.matrizes.trans.lin/con_soma_mult_trans_lin_moodle.html}{espaço vetorial das transformações lineares} de $V$ em $W$. Então a aplicação
		\[\operatorname{L}(V,W)\to M_{m\times n}(\mathbb{R}),\qquad T\mapsto [T]_{\mathcal{B}}^{\mathcal{C}}\]
		é um \href{../14.iso.lin/con_iso_moodle.html}{isomorfismo linear}.
		\item Se $U$ é um outro espaço vetorial com uma base ordenada $\mathcal{D}=\left\{u_1,\ldots,u_l\right\}$, e $T\colon V\to W$ e $S\colon W\to U$ são transformações lineares, então vale que
		\[[S\circ T]_{\mathcal{B}}^{\mathcal{D}}=[S]_{\mathcal{C}}^{\mathcal{D}}[T]_{\mathcal{B}}^{\mathcal{C}},\]
		em que o produto na direita é o produto de matrizes.
	\end{enumerate}
\end{theorem}

\begin{proof}
	\begin{enumerate}
		\item Sejam $T\colon V\to W$ uma transformação linear e $v$ um vetor em $V$. Primeiro, escrevemos $v$ como uma combinação linear dos elementos de $\mathcal{B}$:
		\[v=\lambda_1 v_1+\cdots+\lambda_n v_n,\]
		o que significa que o vetor coordenada de $v$ na base $\mathcal{B}$ é dado por
		\[[v]^{\mathcal{B}}=\begin{bmatrix}\lambda_1\\\vdots\\\lambda_n\end{bmatrix}.\]
		Considere a matriz de $T$ relavia às bases $\mathcal{B}$ e $\mathcal{C}$, $[T]_{\mathcal{B}}^{\mathcal{C}}=[t_{ij}]_{ij}$. Isto significa que para cada $i$ e cada $j$, $t_{ij}$ é a $i$-ésima coordenada de $T(v_j)$.
		
		Assim
		\begin{align*}
			T(v)
				&=T\left(\sum_j\lambda_j v_j\right)\\
				&=\sum_j\lambda_j T(v_j)\\
				&=\sum_j\sum_i\lambda_j t_{ij}w_i\\
				&=\sum_i\left(\sum_j \lambda_j t_{ij}\right)w_i.
		\end{align*}
		Isto significa que a $i$-ésima coordenada de $T(v)$ na base $\mathcal{C}$ é dada por $\sum_j\lambda_j t_{ij}$. Portanto,
		\[[T(v)]^{\mathcal{C}}=\begin{bmatrix}\sum_j\lambda_j t_{1j}\\\vdots\\\sum_j\lambda_j t_{mj}\end{bmatrix}.\tag{1}\]
		Por outro lado,
		\begin{align*}
		[T]_{\mathcal{B}}^{\mathcal{C}}[v]^{\mathcal{B}}
			&=\begin{bmatrix}t_{11}&\cdots&t_{1n}\\\vdots&\ddots&\vdots\\t_{m1}&\cdots&t_{mn}\end{bmatrix}\begin{bmatrix}\lambda_1\\\vdots\\\lambda_n\end{bmatrix}=\begin{bmatrix}\sum_j t_{1j}\lambda_j\\\vdots\\\sum_j t_{mj}\lambda_j\end{bmatrix}.\tag{2}
		\end{align*}
		Como os vetores coluna dos lados direitos das equações (1) e (2) coincidem, isto prova que $[T(v)]^{\mathcal{C}}=[T]_{\mathcal{B}}^{\mathcal{C}}[v]^{\mathcal{B}}$.
		
		\item Seja $A\in M_{m\times n}(\mathbb{R})$ uma matriz $m\times n$. Para construir uma transformação linear $T$ cuja representação respectiva às bases $\mathcal{B}$ e $\mathcal{C}$ seja dada por $A$, notemos o seguinte: Caso tenhamos tal $T$, então a $j$-ésima coluna de $A$ será o vetor coluna das coordenadas de $T(v_j)$ com relação à base $\mathcal{C}$, o que significa que procuramos $T\colon V\to W$ tal que
		\[T(v_j)=\sum_i a_{ij}w_i\qquad\text{para todo }j.\tag{3}\]
		Como transformações lineares estão unicamente determinadas por seus valores na base $\mathcal{B}$, existe exatamente uma transformação linear $T\colon V\to W$ satisfazendo (3), ou equivalentemente que $[T]_{\mathcal{B}}^{\mathcal{C}}=A$.
		
		\item A aplicação $T\mapsto [T]_{\mathcal{B}}^{\mathcal{C}}$ é bijetiva pelo item anterior. A verificação de que esta aplicação é linear (e portanto um isomorfismo linear) é rotineira: Se $T_1,T_2\in\operatorname{L}(V,W)$ e $\lambda\in\mathbb{R}$, então temos que
		\begin{align*}
			[T_1+\lambda T_2]_{\mathcal{B}}^{\mathcal{C}}
				&\overset{(1)}{=}\begin{bmatrix}|&&|\\ [(T_1+\lambda T_2)(v_1)]^{\mathcal{C}}&\cdots&[(T_1+\lambda T_2)(v_n)]^{\mathcal{C}}\\|&&|\end{bmatrix}\\
				&\overset{(2)}{=}\begin{bmatrix}|&&|\\ [T_1(v_1)+\lambda T_2(v_1)]^{\mathcal{C}}&\cdots&[T_1(v_n)+\lambda T_2(v_n)]^{\mathcal{C}}\\|&&|\end{bmatrix}\\
				&\overset{(3)}{=}\begin{bmatrix}|&&|\\ [T_1(v_1)]^{\mathcal{C}}+\lambda [T_2(v_1)]^{\mathcal{C}}&\cdots&[T_1(v_n)]^{\mathcal{C}}+\lambda[T_2(v_n)]^{\mathcal{C}}\\|&&|\end{bmatrix}\\
				&\overset{(4)}{=}\begin{bmatrix}|&&|\\ [T_1(v_1)]^{\mathcal{C}}&\cdots&[T_1(v_n)]^{\mathcal{C}}\\|&&|\end{bmatrix}+\lambda\begin{bmatrix}|&&|\\ [T_2(v_1)]^{\mathcal{C}}&\cdots&[T_2(v_n)]^{\mathcal{C}}\\|&&|\end{bmatrix}\\
				&\overset{(5)}{=}[T_1]_{\mathcal{B}}^{\mathcal{C}}+\lambda[T_2]_{\mathcal{B}}^{\mathcal{C}}.
		\end{align*}
		onde
		\begin{itemize}
			\item (1) é a definição de matriz associada a uma transformação linear;
			\item (2) é a definição de soma e multiplicação por escalar para transformações lineares;
			\item (3) segue do fato que a função de $W$ a $M_{m\times 1}(\mathbb{R})$, que associa a cada vetor $w$ seu vetor coordenada $[w]^{\mathcal{C}}$ na base $\mathcal{C}$, é linear;
			\item (4) é a definição de soma e multiplicação por escalar para matrizes;
			\item (5) é novamente a definição de matrizes associadas a transformações lineares.
		\end{itemize}
		
		Isto prova que a função $T\mapsto [T]_{\mathcal{B}}^{\mathcal{C}}$ é linear.
		
		\item Sejam $S$ e $T$ como no enunciado, com as respectivas matrizes $[S]_{\mathcal{C}}^{\mathcal{D}}=[s_{li}]_{li}$ e $[T]_{\mathcal{B}}^{\mathcal{C}}=[t_{ij}]_{ij}$. Vamos calcular a $j$-ésima coluna de $[S\circ T]_{\mathcal{B}}^{\mathcal{D}}$.
		\begin{align*}
			(S\circ T)(v_j)
				&=S(T(v_j))\\
				&=S\left(\sum_i t_{ij}w_i\right)\\
				&=\sum_i t_{ij}S(w_i)\\
				&=\sum_it_{ij}\sum_ks_{ki}u_k\\
				&=\sum_k\left(\sum_is_{ki}t_{ij}\right)u_k.
		\end{align*}
		Isto significa que a entrada $(k,j)$ da matriz $[S\circ T]_{\mathcal{B}}^{\mathcal{D}}$ é dada por $\sum_is_{ki}t_{ij}$. Este é exatamente o valor da mesma entrada $(k,j)$ da matriz produto $[s_{ki}]_{ki}[t_{ij}]_{ij}=[S]_{\mathcal{C}}^{\mathcal{D}}[T]_{\mathcal{B}}^{\mathcal{C}}$.
		
		Concluímos que as matrizes $[S\circ T]_{\mathcal{B}}^{\mathcal{D}}$ e $[S]_{\mathcal{C}}^{\mathcal{D}}[T]_{\mathcal{B}}^{\mathcal{C}}$ são iguais
	\end{enumerate}
\end{proof}