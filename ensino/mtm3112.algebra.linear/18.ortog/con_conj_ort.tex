\begin{definition}
	Seja $V$ um espaço vetorial com produto interno e seja $A\subseteq V$ um subconjunto de $V$.
	
	Dizemos que $A$ é \textbf{ortogonal} se os elementos de $A$ são dois-a-dois ortogonais. Mais explicitamente, requeremos que $\langle a,\widetilde{a}\rangle=0$ sempre que $a,\widetilde{a}\in A$ e $a\neq\widetilde{a}$.
	
	Dizemos que $A$ é \textbf{ortonormal} $A$ for ortogonal e, além disso, seus elementos têm norma igual a $1$ (também se diz que eles são \textit{unitários}). Mais explicitamente, requeremos que para todos $a,\widetilde{a}\in A$,
	\[\langle a,\widetilde{a}\rangle=\begin{cases}1, &\text{se }a=\widetilde{a}\\0,&\text{se }a\neq \widetilde{a}.\end{cases}\]
\end{definition}