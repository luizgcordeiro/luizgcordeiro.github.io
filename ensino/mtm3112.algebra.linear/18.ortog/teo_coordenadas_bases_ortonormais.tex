\begin{theorem}
	Sejam $V$ um espaço com produto interno e $\mathcal{B}$ uma base ortonormal para $V$. Então para cada vetor $v\in V$ e cada vetor $b\in\mathcal{B}$, a $b$-ésima coordenada de $v$ na base $\mathcal{B}$ é dada por $\langle v,b\rangle$. Em outros termos, cada vetor $v\in V$ é representado na base $\mathcal{B}$ do modo
	\[v=\sum_{b\in\mathcal{B}}\langle v,b\rangle b.\]
\end{theorem}

\begin{proof}
	Dado $v\in V$, representamos
	\[v=\sum_{x\in\mathcal{B}}\alpha_x x\tag{$*$}\]
	(onde somente uma quantidade finita dos coeficients $\alpha_b$ são não-nulos).
	
	Basta então provar que $\alpha_b=\langle v,b\rangle$ para cada $b\in\mathcal{B}$. Vamos tomar produtos escalares com $b$ à direita em ambos os lados da equação ($*$).
	\begin{align*}
		\langle v,b\rangle
			&=\left\langle \sum_{x\in\mathcal{B}}\alpha_x x,b\right\rangle\\
			&=\sum_{x\in\mathcal{B}}\alpha_x\langle x,b\rangle,\tag{$**$}
	\end{align*}
	onde a segunda igualdade é obtida pela linearidade do produto interno na primeira entrada. Como $\mathcal{B}$ é uma base ortonormal contendo $b$, então para cada $x\in\mathcal{B}$ temos que
	\[\langle x,b\rangle=\begin{cases}1,&\text{se } x=b\\0,&\text{se }x\neq b.\end{cases}\]
	Isto significa que o único termo não-nulo na soma em ($**$) é dado pelo índice $x=b$. Portanto ($**$) se reescreve como
	\[\langle v,b\rangle=\alpha_b\langle b,b\rangle=\alpha_b,\]
	exatamente como queríamos provar.
\end{proof}