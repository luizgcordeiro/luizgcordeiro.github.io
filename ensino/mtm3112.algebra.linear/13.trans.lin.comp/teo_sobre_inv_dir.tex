\begin{theorem}
	Uma tranformação linear $T\colon V\to W$ é sobrejetiva se, e somente se, $T$ admite uma inversa linear à direita, i.e., se existe uma transformação linear $S\colon W\to V$ tal que $T\circ S=\operatorname{id}_W$.
\end{theorem}

\begin{proof}
	Lembre-se do seguinte fato geral: Se $f\colon X\to Y$ é uma função entre dois conjuntos, então $f$ é sobrejetiva se, e somente se, $f$ admite uma inversa à direita. Assim, se $T$ admite uma inversa linear à direita então $T$ é injetiva. O problema deste teorema se resume então a provar que, no caso de $T$ ser sobrejetiva, pode-se tomar uma inversa à direita de $T$ que seja linear.
	
	Suponhamos que $T$ seja sobrejetiva. Considere uma base qualquer $\mathcal{C}$ de $W$. Para definir uma função linear $S\colon W\to V$, basta definir $S$ na base $\mathcal{C}$. Como $T$ é sobrejetiva, então para cada $c\in\mathcal{C}$ podemos escolher um vetor $S(c)\in V$ tal que $T(S(c))=c$.
	
	Assim, temos $S$ definida em $\mathcal{C}$, que se estende linearmente para uma função linear de $W$ em $V$.
	
	Se $c\in\mathcal{C}$, então $T(S(c))=c$, o que mostra que $T\circ S$ é a identidade em $\mathcal{C}$. Como $\mathcal{C}$ é uma base de $W$, então $T\circ S$ é a identidade de $W$.
\end{proof}