\begin{theorem}
	Uma tranformação linear $T\colon V\to W$ é injetiva se, e somente se, $T$ admite uma inversa linear à esquerda, i.e., se existe uma transformação linear $S\colon W\to V$ tal que $S\circ T=\operatorname{id}_V$.
\end{theorem}

\begin{proof}
	Lembre-se do seguinte fato geral: Se $f\colon X\to Y$ é uma função entre dois conjuntos, então $f$ é injetiva se, e somente se, $f$ admite uma inversa à esquerda. Assim, se $T$ admite uma inversa linear à esquerda então $T$ é injetiva. O problema deste teorema se resume então a provar que, no caso de $T$ ser injetiva, pode-se tomar uma inversa à esquerda de $T$ que seja linear.
	
	Suponhamos que $T$ seja injetiva. Considere uma base qualquer $\mathcal{B}$ de $V$. Então o conjunto imagem direta
	\[T(\mathcal{B})=\left\{T(b):b\in\mathcal{B}\right\}\]
	de $\mathcal{B}$ por $T$ é linearmente independente em $W$ (visto que $T$ é injetiva). Segue que existe uma base $\mathcal{C}$ de $W$ tal que $T(\mathcal{B})\subseteq\mathcal{C}$.
	
	Para definir uma função linear $S\colon W\to V$, basta definir $S$ na base $\mathcal{C}$. Dado $c\in\mathcal{C}$, temos dois casos:
	\begin{itemize}
		\item Se $c\in T(\mathcal{B})$, então $c=T(b)$ para algum $b\in\mathcal{B}$, e este $b$ e único pois $T$ é injetiva. Definimos $S(c)=b$.
		\item Caso $c\not\in T(\mathcal{B})$, defina $S(c)=0$.
	\end{itemize}
	
	Então $S(c)$ se estende linearmente para uma transformação linear de $W$ em $V$.
	
	Se $b\in\mathcal{B}$, então $S(T(b))=b$, o que mostra que $S\circ T$ é a identidade em $\mathcal{B}$. Como $\mathcal{B}$ é uma base de $V$, então $S\circ T$ é a identidade de $V$.
\end{proof}