\begin{definition}
	Um \textbf{subespaço vetorial} de um espaço vetorial $V$ é um subconjunto não-vazio $W$ satisfazendo às seguintes propriedades:
	
	\begin{enumerate}
		\item Para quaisquer dois vetores $x,y\in W$, temos que $x+y\in W$;
		\item Para todo vetor $x\in W$ e para todo escalar $\lambda\in\mathbb{R}$, temos que $\lambda x\in W$.
	\end{enumerate}
	
	Em outras palavras, ao realizarmos as operações de espaço vetorial (soma de vetores e multiplicação escalar) com elementos de $W$, o vetor resultante ainda será um elemento de $W$. (Diz-se que $W$ é \textit{fechado} com relação às operações.)
	
	\hrule
	
	\textit{Observação}: Às vezes se adiciona o axioma de que ``$0_V\in W$'', porém este fato decorre dos dois axiomas acima: Como $W$ é não-vazio, podemos tomar um elemento $w\in W$ qualquer. Pelo axioma 2, teremos que o produto $0w=0_V$ também pertence a $W$.
\end{definition}