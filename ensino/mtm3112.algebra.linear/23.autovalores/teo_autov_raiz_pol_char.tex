\begin{theorem}
	Seja $T\colon V\to V$ um endomorfismo linear em um espaço vetorial $V$ de dimensão finita.
	
	Então os autovalores de $T$ são exatamente as raízes do polinômio característico $p_T(\lambda)$.
\end{theorem}

\begin{proof}
	Primeiro, vamos supor que $\lambda$ é um autovalor de $T$. Vamos mostrar que $\lambda$ é uma raiz de $p_T(\lambda)$.
	
	De fato, a definição de autovalor nos diz que existe um vetor não nulo $v\neq 0_V$ (um autovetor associado) tal que $T(v)=\lambda v$. Equivalentemente, isto significa que
	\[(\lambda\operatorname{id}_V-T)(v)=0_V.\]
	Isto significa que $v$ pertence ao núcleo de $\lamdba\operatorname{id}_V-T$, que portanto não é trivial. Logo, $\lambda\operatorname{id}_V-T$ não é injetiva, e como estamos considerando um espaço de dimensão finita, isto significa que $\lambda\operatorname{id}_V-T$ não é inversível.
	
	Agora, lembremos do seguinte fato acerca do determinante: Uma matriz é inversível se e somente se o seu determinante é não-nulo. O mesmo fato é válido para endomorfismos lineares (a prova pode ser feita utilizando o teorema acerca da correspondência entre matrizes e transformações lineares, mas vamos omití-la).
	
	Portanto, como $\lambda\operatorname{id}_V-T$ é não inversível então seu determinante é nulo, ou seja,
	\begin{align*}
	p_T(\lambda)
		&=\det\left(\lambda\operatorname{id}_V-T\right)\\
		&=0.
	\end{align*}
	o que significa que $\lamdba$ é, de fato, uma raiz do polinômio característico $p_T$ de $T$.
	
	Assim, demonstramos que todo autovalor de $T$ é uma raiz do seu polinômio característico.
	
	\hrule
	
	A prova da recíproca -- em que se mostra que toda raiz do polinômio característico é autovalor -- é feita utilizando argumentos similares, e vamos deixá-la como exercício.
\end{proof}