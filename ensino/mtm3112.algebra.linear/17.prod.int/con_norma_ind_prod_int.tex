\begin{definition}
	Seja $V$ um espaço vetorial com um produto interno $\langle\cdot,\cdot\rangle$. A \textbf{norma induzida do produto interno} é a norma $\Vert\cdot\Vert_{\langle\cdot,\cdot,\rangle}$ definida por
	\[\Vert v\Vert_{\langle\cdot,\cdot\rangle}=\sqrt{\langle v,v\rangle}\]
	para todo vetor $v\in V$.
	
	\hrule
	
	\textit{Observações}:
	\begin{itemize}
		\item A verificação de que $\Vert\cdot\Vert_{\langle\cdot,\cdot\rangle}$ é de fato uma norma é feita em um teorema.
		\item No restante do curso, e sempre que não houver risco de ambiguidade, nós sempre consideraremos espaços vetoriais com produtos internos bem-especificados, e somente a norma induzida por este produto interno. Assim, poderemos simplificar a notação, e escrever somente ``$\Vert\cdot\Vert$'' para a norma induzida por um produto interno.
	\end{itemize}
\end{definition}