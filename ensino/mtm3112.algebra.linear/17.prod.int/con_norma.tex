\begin{definition}
	Uma \textbf{norma} em um espaço vetorial real \(V\) é uma função
	\[\begin{array}{r c c c}
		\Vert\cdot\Vert\colon
			&V
				&\longrightarrow
					&[0,+\infty)\\
			&v
				&\longmapsto
					&\Vert v\Vert
		\end{array}\]
	satisfazendo às seguintes condições:
	
	\begin{enumerate}
		\item \uline{Subaditividade}/\uline{desigualdade triangular}: Para todos $u,v\in V$ vale que
			\[\Vert u+v\Vert\leq\Vert u\Vert+\Vert v\Vert.\]
		\item \uline{Homogeneidade em módulo}: Para todos $v\in V$ e todo $\lambda\in\mathbb{R}$ vale que
			\[\Vert\lambda u\Vert=|\lambda|\Vert u\Vert.\]
		\item \uline{Positividade estrita}/\uline{não-degenerecência}: Se $u\neq 0_V$, então $\Vert u\Vert>0$.
	\end{enumerate}
	
	\hrule
	
	\textit{Observação}: Neste curso, só consideraremos normas induzidas de produtos internos.
\end{definition}