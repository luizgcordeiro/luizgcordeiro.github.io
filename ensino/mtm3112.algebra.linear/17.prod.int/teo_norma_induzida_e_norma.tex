\begin{theorem}
	Seja $V$ um espaço vetorial munido com um produto interno $\langle\,\cdot\,,\,\cdot\,\rangle$. Então a expressão
	\[\Vert v\Vert=\sqrt{\langle v,v\rangle}\]
	define uma norma em $V$.
\end{theorem}

\begin{proof}
	A única propriedade não-trivial que se deve verificar é a desigualdade triangular. Sejam $u,v\in V$. Então
	\begin{align*}
	\Vert u+v\Vert^2
		&=\langle u+v,u+v\rangle\\
		&=\langle u,u\rangle+2\langle u,v\rangle+\langle v,v\rangle\\
		&=\Vert u\Vert^2+2\langle u,v\rangle+\Vert v\Vert^2.
	\end{align*}
	Pela Desigualdade de Cauchy--Bunyakovsky--Schwarz, temos que $\langle u,v\rangle\leq\Vert u\Vert \Vert v\Vert$, e portanto
	\begin{align*}
	\Vert u+v\Vert^2
		&\leq\Vert u\Vert^2+2\Vert u\Vert\Vert v\Vert+\Vert v\Vert^2\\
		&=\left(\Vert u\Vert+\Vert v\Vert\right)^2.
	\end{align*}
	
	Tomando raízes quadradas em ambos os lados da equação acima (o que é permitido pois todos os termos de interesse são não-negativos), concluímos que
	\[\Vert u+v\Vert\leq\Vert u\Vert+\Vert v\Vert,\]
	precisamente a desigualdade triangular.
	
	A prova de que as outras propriedades de norma são satisfeitas é deixada como um exercício.
\end{proof}