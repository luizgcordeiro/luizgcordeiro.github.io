\begin{definition}
	O \textbf{produto interno Euclidiano} de $\mathbb{R}^n$ (também chamado de \textbf{produto interno padrão}) é o produto interno em $\mathbb{R}^n$ dado por
	\begin{align*}
		\langle (x_1,\ldots,x_n),(y_1,\ldots,y_n)\rangle
			&=\sum_{i=1}^n x_iy_i\\
			&=x_1y_1+\cdots+x_ny_n
	\end{align*}
	para todos os vetores $(x_1,\ldots,x_n),(y_1,\ldots,y_n)$ de $\mathbb{R}^n$.
	
	\hrule
	
	Caso denotemos os vetores de $\mathbb{R}^n$ como matrizes coluna, o produto interno Euclidiano também é comumente denotado utilizando produto de matrizes: Dados vetores
	\[\mathbf{x}=\begin{bmatrix}x_1\\\vdots\\x_n\end{bmatrix}\quad\text{e}\quad\mathbf{y}=\begin{bmatrix}y_1\\\vdots\\y_n\end{bmatrix},\]
	temos
	\[\langle\mathbf{x},\mathbf{y}\rangle=\mathbf{y}^T\mathbf{x}.\]
\end{definition}