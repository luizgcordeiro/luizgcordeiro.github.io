\begin{definition}
	Um \textbf{endomorfismo linear} é uma transformação linear de um espaço vetorial $V$ a si mesmo. Em outros termos, chamamos uma transformação linear $T\colon V\to V$, cujo domínio e contradomínio são iguais de um \textbf{endomorfismo linear}.
\end{definition}

\hrule

\textit{Observação}: A definição acima serve simplesmente para especificar a convenção que iremos adotar durante o curso. Às vezes, chama-se de \textit{operador} linear uma transformação liner com domínio e contradomínio iguais. Mas às vezes se utiliza \textit{operador linear} como sinônimo de \textit{transformação linear}, sem qualquer restrição sobre domínio e contradomínio. O termo \textit{endomorfismo} é inequívoco, e por isso o adotamos.