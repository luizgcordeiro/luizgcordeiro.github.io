\begin{theorem}
	Considere matrizes diagonais $A$ e $B$ de mesma ordem:
	\begin{align*}A&=\begin{bmatrix}
		a_{1}&0&\cdots&0\\
		0&a_{2}&\cdots&0\\
		\vdots&\vdots&\ddots&\vdots
		0&0&\cdots&a_{n}
		\end{bmatrix},\\
		B&=\begin{bmatrix}
		b_{1}&0&\cdots&0\\
		0&b_{2}&\cdots&0\\
		\vdots&\vdots&\ddots&\vdots
		0&0&\cdots&b_{n}
		\end{bmatrix}.\end{align*}
	Então
	\begin{enumerate}
		\item $A+\lambda B=\begin{bmatrix}
		a_1+\lambda b_{1}&0&\cdots&0\\
		0&a_2+\lambda b_{2}&\cdots&0\\
		\vdots&\vdots&\ddots&\vdots
		0&0&\cdots&a_n+\lambda b_{n}
		\end{bmatrix}$, para todo $\lambda\in\mathbb{R}$.
		\item $AB=\begin{bmatrix}
		a_1b_{1}&0&\cdots&0\\
		0&a_2b_{2}&\cdots&0\\
		\vdots&\vdots&\ddots&\vdots
		0&0&\cdots&a_nb_{n}
		\end{bmatrix}$.
		\item Seja $\operatorname{Diag}_n(\mathbb{R})$ o subespaço de $M_n(\mathbb{R})$ consistindo das matrizes diagonais de ordem $n\times n$. Então a função
		\[\operatorname{diag}\colon\mathbb{R}^n\to\operatorname{Diag}_n(\mathbb{R})\]
		dada por
		\[\operatorname{diag}(a_1,\ldots,a_n)=\begin{bmatrix}a_1&0&\cdots&0\\
		0&a_2&\cdots&0\\\vdots&\vdots&\ddots&\vdots\\0&0&\cdots&a_n\end{bmatrix}\]
		é um isomorfismo linear.
	\end{enumerate}
\end{theorem}

\begin{proof}
	Os itens 1 e 2 são imediatos das definições das operações com matrizes. A linearidade da função $\operatorname{diag}$ no item 3 segue do item 1, e é claro que $\operatorname{diag}$ é bijetiva, logo um isomorfismo linear.
\end{proof}