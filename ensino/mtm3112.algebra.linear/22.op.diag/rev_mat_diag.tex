Uma matriz quadrada $A\in M_{n\times n}(\mathbb{R})$ é dita ser \textbf{diagonal} se as entradas não-nulas de $A$ se encontram em sua diagonal principal. Isto significa que se $A=[a_{ij}]_{i,j}$, então $a_{ij}=0$ quando $i\neq j$ e $A$ tem a forma
\[A=\begin{bmatrix}
a_{11} & 0 & \cdots & 0 & 0\\
0 & a_{22} & \cdots & 0 & 0\\
\vdots & \vdots & \ddots & \vdots & \vdots\\
0 & 0 & \cdots & a_{n-1,n-1} & 0\\
0 & ) & \cdots * 0 & a_{nn}\end{bmatrix}.\]