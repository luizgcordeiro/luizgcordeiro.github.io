Seja $A$ uma matriz complexa Hermitiana de ordem $n\times n$. Vamos descrever um algoritmo para diagonalizar $A$. Isto significa que conseguiremos encontrar uma matriz unitária $P$ tal que $P^HAP$ é uma matriz diagonal.

\begin{enumerate}
	\item Primeiro, calcule o polinômio característico de $A$, como
		\[p_A(\lambda)=\det(\lambda I_n-A).\]
	\item Então, calcule as raízes de $p_A$. Digamos que $\lambda_1,\ldots,\lambda_k$ são estas raízes. Estas raízes são os autovalores de $A$.
	\item Para cada $i=1,\ldots,k$, encontre uma base para $\operatorname{Nul}(\lambda_i I_n-A)$ -- o autoespaço de $A$ associado ao autovalor $\lambda_i$ -- utilizando o Algoritmo para Encontrar Base para Espaço Nulo de uma Matriz.
	
	Seja $\left\{e_1^i,\ldots,e_{n_i}^i\right\}$ a base obtida por este processo.
	\item Aplique o Processo de Ortonormalização de Gram-Schmidt para transformar cada base $\left\{e_1^i,\ldots,e_{n_i}^i\right\}$ em uma base ortonormal para $\operatorname{Nul}(\lambda_i I_n-A)$. Seja $\left\{f_1^i,\ldots,f_{n_i}^i\right\}$ a base ortonormal obtida.
	\item Agora, considere a matriz $P$ cujas colunas são todos os vetores $f_j^i$.
\end{enumerate}

A matriz $P$ obtida no passo 5 será tal que $P^H A P$ é diagonal. As entradas diagonais de $P^{-1}AP$ serão $\lambda_1,\ldots,\lambda_k$ (possivelmente com repetição).

No passo 5, os vetores $f_j^i$ podem ser postos em qualquer ordem como as colunas da matriz $P$, mas em geral é mais interessante que estes sejam dispostos de forma organizada para agrupar as entradas diagonais iguais de $P^{-1}AP$. Mais precisamente, suponha que as colunas de $P$ são dispostas em ordem como $f^1_1,\ldots,f^1_{n_1}, f^2_1,\ldots,f^2_{n_2},f^3_1,\ldots,f^3_{n_3},\ldots,f^k_1,\ldots,f^k_{n_k}$:
\[
P=\begin{bmatrix}
|	&|	&|	&|	&|	&|	&|	&|	&|	&|\\
f^1_1
	&\cdots
		&f^1_{n_1}
			&f^2_1
				&\cdots
					&f^2_{n_2}
						&\cdots
							&f^k_1
								&\cdots
									&f^k_{n_k}\\
|	&|	&|	&|	&|	&|	&|	&|	&|	&|
\end{bmatrix}.\]

Então a matriz $P^HAP$ terá suas entradas diagonais agrupadas, na forma
\[P^HAP=\begin{bmatrix}
\lambda_1\\
	&\ddots\\
	&	&\lambda_1\\
	&	&	&\lambda_2\\
	&	&	&	&\ddots\\
	&	&	&	&	&\lambda_2\\
	&	&	&	&	&   &\ddots\\
	&	&	&	&	&	&   &\lambda_k\end{bmatrix}.\]