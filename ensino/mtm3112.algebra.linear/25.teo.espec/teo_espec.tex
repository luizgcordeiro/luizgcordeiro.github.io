\begin{theorem}[Teorema Espectral]
	Seja $V$ um espaço complexo com produto interno e dimensão finita. Seja $T\colon V\to V$ um operator auto-adjunto de $V$.
	
	Então:
	\begin{enumerate}
		\item Todos os autovalores de $T$ são reais.
		\item Autovalores que correspondem a autovetores distintos são ortogonais.
		\item Existe uma base ortonomal de $V$ consistindo de autovetores de $T$.
	\end{enumerate}
\end{theorem}

\begin{proof}
	Como $T$ é auto-adjunto, então
	\[\langle T(v),w\rangle=\langle v,T(w)\rangle.\tag{AA}\]
	para todos os vetores $v,w\in V$.
	\begin{enumerate}
		\item Seja $\lambda$ um autovalor de $T$ com autovetor correspondente $v_\lambda$. Então utilizamos a igualdade (AA) com $v=w=v_\lambda$, e obtemos
		\begin{align*}
			\langle Tv_\lambda,v_\lambda\rangle
				&=\langle v_\lambda,Tv_\lambda\rangle\\
			\langle \lambda v_\lambda,v_\lambda\rangle
				&=\langle v_\lambda,\lambda v_\lambda\rangle\\
			\lambda \langle v_\lambda,v_\lambda\rangle
				&=\overline{\lambda}\langle v_\lambda,v_\lambda\rangle.
		\end{align*}
		Como $v_\lambda\neq 0_V$ (pois $v_\lambda$ é um autovetor), então $\lambda=\overline{\lambda}$, logo $\lambda$ é real.
		
		\item Sejam $\lambda_1$ e $\lambda_2$ dois autovalores distintos de $T$, com autovetores respectivos $v_1$ e $v_2$. Vamos utilizar a igualdade (AA) com $v=v_1$ e $w=v_2$.
		\begin{align*}
			\langle Tv_1,v_2\rangle&=\langle v_1,Tv_2\rangle\\
			\langle \lambda_1 v_1,v_2\rangle&=\langle v_1,\lambda_2 v_2\rangle\\
			\lambda_1 \langle v_1,v_2\rangle&=\overline{\lambda_2}\langle v_1,v_2\rangle\\
			\lambda_1 \langle v_1,v_2\rangle&=\lambda_2\langle v_1,v_2\rangle,
		\end{align*}
		onde a última igualdade é obtida da anterior pelo primeiro item, pois $\lambda_2$ é real. Como $\lambda_1\neq\lambda_2$, a última igualdade implica que $\langle v_1,v_2\rangle=0$, como queríamos.
		
		\item Seja $\lambda_1$ um autovalor qualquer de $T$ com um autovetor $u_1$. Substituindo $u_1$ por $\frac{u_1}{\Vert u_1\Vert}$, podemos assumir que $u_1$ tem norma $1$. Se $\dim(V)=1$, basta considerar a base $\left\{u_1\right\}$ de $V$, e concluímos o teorema. Vamos então assumir que $\dim(V)\geq 2$.
		
		Considere o complemento ortogonal $V_2=\langle u_1\rangle^\perp$. Este é o subespaço de $V$ consistindo dos vetores ortogonais a $u_1$.
		
		Vamos provar que $V_2$ é invariante por $T$, i.e., que $T(V_2)\subseteq V_2$.
		
		\begin{quote}
			De fato, se $v\in V_2$ então $v$ é ortogonal a $u_1$. Uamos a igualdade (AA) com $w=u_1$ para obter
			\begin{align*}
				\langle Tv,u_1\rangle
					&=\langle v,Tu_1\rangle\\
					&=\langle v,\lambda_1 u_1\rangle\\
					&=\overline{\lambda_1}\langle v,u_1\rangle\\
					&=0,
			\end{align*}\]
			logo $T(v)$ também é ortogonal a $u_1$, ou seja, $T(v)\in V_2$.
		\end{quote}
		
		Portanto, podemos considerar a restrição de $T$ ao subespaço $V_2$, que é um operador $T|_{V_2}\colon V_2\to V_2$. Além disso, a igualdade (AA) também é válida quando restrita para vetores de $V_2$, o que significa que $T|_{V_2}$ é auto-adjunto (como operador em $V_2$). Note que $\dim(V_2)=\dim(V)-1$.

		Este passo nos permite reduzir a prova do teorema a um subespaço de dimensão menor do que a original. Então repetindo este procedimento, iremos reduzir o teorema para subespaços de dimensão $\dim(V)-1$, $\dim(V)-2$, $\dim(V)-3$, \ldots, até dimensão $1$, um caso que já foi resolvido.
		
		Mais formalmente, seja $\lambda_2$ um autovalor de $T|_{V_2}$, com autovetor correspondente $u_2$. Novamente substituindo $u_2$ por $\frac{u_2}{\Vert u_2\Vert}$ caso necessário, podemos assumir que $\Vert u_2\Vert=1$. Claramente $\lambda_2$ também é um autovalor de $T$, e $u_2$, visto como vetor de $V$, é um autovetor correspondente. Além disso, $u_2$ é ortogonal a $u_1$, por construção. Assim, obtemos autovetores $u_1$ e $u_2$ de $T$ que são ortonormais. Se $\dim(V)=2$, terminamos o procedimento aqui, considerando a base $\left\{u_1,u_2\right\}$.
		
		Caso $\dim(V)\geq 3$, consideremos os subespaço $V_3=\langle u_1,u_2\rangle^\perp$ dos vetores ortogonais a $u_1$ e $u_2$. Note que $\dim(V_3)=\dim(V)-2$. O mesmo argumento ao realizado acima permite concluir que $T(V_3)\subseteq V_3$, e assim obtemos um operador restrito $T|_{V_3}\colon V_3\to V_3$. Novamente, este operador é auto-adjunto em $V_3$. Tome um autovalor $\lambda_3$ de $T|_{V_3}$ com um autovetor associado $u_3$. Novamente, $\lambda_3$ e $u_3$ também são autovalor e autovetor de $T$, e $u_3$ é ortogonal a $u_1$ e a $u_2$ por construção. Assim, obtemos uma coleção ortonormal $\left\{u_1,u_2,u_3\right\}$ de autovetores de $T$. Se $\dim(V)=3$, terminamos a prova.
		
		Repetimos este procedimento até encontrar uma família ortonormal $\left\{u_1,\ldots,u_n\right\}$ de autovetores de $T$, onde $n=\dim(V)$. Em particular, $\left\{u_1,\ldots,u_n\right\}$ é linearmente independente, e seu número de elementos é a dimensão do espaço $V$. Portanto, $\left\{u_1,\ldots,u_n\right\}$ é de fato uma base ortonormal para $V$.
	\end{enumerate}
	
\end{proof}